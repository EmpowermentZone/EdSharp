%%%%%%%%%%%%%%%%%%%%%%%%%%%%%%%%%%%%%%%%%%%%%%%%%%%%%%%%%%%%%%%%%
% Contents: Who contributed to this Document
% $Id: overview.tex,v 1.6 1998/09/29 08:05:09 oetiker Exp oetiker $
%%%%%%%%%%%%%%%%%%%%%%%%%%%%%%%%%%%%%%%%%%%%%%%%%%%%%%%%%%%%%%%%%
\chapter{Preface}

\LaTeX{} \cite{manual} is a typesetting system which is very 
suitable for producing scientific and mathematical documents of high
typographical quality. The system is also suitable for producing all
sorts of other documents, from simple letters to complete books.
\LaTeX{} uses \TeX{} \cite{texbook} as its formatting engine.

This short introduction describes \LaTeXe{} and should be sufficient
for most applications of \LaTeX. Refer to~\cite{manual,companion} for
a complete description of the \LaTeX{} system.

\LaTeX{} is available for most computers, from the PC and Mac to large
UNIX and VMS systems. On many university computer clusters, you will
find that a \LaTeX{} installation is available, ready to use.
Information on how to access
the local \LaTeX{} installation should be provided in the \guide. If
you have problems getting started, ask the person who gave you this
booklet. The scope of this document is \emph{not} to tell you how to
install and set up a \LaTeX{} system, but to teach you how to write
your documents so that they can be processed by~\LaTeX{}.

\noindent This Introduction is split into 5 chapters:
\begin{description}
\item[Chapter 1] tells you about the basic structure of \LaTeXe{}
  documents. You will also learn a bit about the history of \LaTeX{}.
  After reading this chapter, you should have a rough picture of
  \LaTeX{}. The picture will only be a framework, but it will enable
  you to integrate the information provided in the other chapters into
  the big picture.
\item[Chapter 2] goes into the details of typesetting your
  documents. It explains most of the essential \LaTeX{} commands and
  environments. After reading this chapter, you will be able to write
  your first documents. 
\item[Chapter 3] explains how to typeset formulae with \LaTeX. Again, a
  lot of examples help you to understand how to use one of \LaTeX{}'s
  main strengths. At the end of this chapter, you will find tables, listing
  all the mathematical symbols available in \LaTeX{}.
\item[Chapter 4] explains index and  bibliography generation,
  inclusion of EPS graphics, and some other useful extensions.
\item[Chapter 5] contains some potentially dangerous information about
  how to make alterations to the
  standard document layout produced by \LaTeX{}. It will tell you how  to
  change things such that the beautiful output of \LaTeX{}
  begins looking quite bad.
\end{description}
\bigskip
It is important to read the chapters in sequential order. The book is
not that big after all. Make sure to carefully read the examples,
because a great part of the information is contained in the various
examples you will find all throughout the book.

\bigskip
\noindent If you need to get hold of any \LaTeX{} related material, 
have a look in one of the \texttt{CTAN} ftp archives. They can be
found e.g.{} at \texttt{ctan.tug.org} (US), \texttt{ftp.dante.de} (Germany), \texttt{ftp.tex.ac.uk}
(UK). If you are not in one of these countries, choose the archive
closest to you.

If you want to run \LaTeX{} on your own computer, take a look at what
is available from \texttt{CTAN:/tex-archive/systems}.

\vspace{\stretch{1}}
\noindent If you have ideas for something to be
added, removed or altered in this document, please let me know. I am
especially interested in feedback from \LaTeX{} novices about which
bits of this intro are easy to understand and which could be explained
better.

\bigskip
\begin{verse}
\contrib{Tobias Oetiker}{oetiker@ee.ethz.ch}%
\noindent{Department of Electrical Engineering,\\
Swiss Federal Institute of Technology}
\end{verse}
\vspace{\stretch{1}}
\noindent The current version of this document is available on\\
\texttt{CTAN:/tex-archive/info/lshort}

\endinput



%%% Local Variables: 
%%% mode: latex
%%% TeX-master: "lshort2e"
%%% End: 
