\documentclass[11pt]{article}  % required first line, though can vary;
                               % this says we will use 11-point font,
                               % in the "article" format

% material beginning with the percent sign is commentary, for human
% information purposes, not processed by the LaTeX system

% these \setlength lines concern page layout, amount of paragraph
% indentation etc.; beginners should ignore them (but include them)
\setlength{\oddsidemargin}{0.0in}
\setlength{\evensidemargin}{0.0in}
\setlength{\topmargin}{-0.25in}
\setlength{\headheight}{0in}
\setlength{\headsep}{0in}
\setlength{\textwidth}{6.5in}
\setlength{\textheight}{9.25in}
\setlength{\parindent}{0in}
\setlength{\parskip}{2mm}

\begin{document}  % required; doc starts here

The famous Pythagoren Theorm concerns right triangles.  If the legs of
the triangle have lengths x and y, and the length of the hypotenus is z,
% the $ delimiter marks the start and end of a mathematical expression
then $z = \sqrt{x^2+y^2}$.

% a blank line means a new paragraph

There are many proofs of this result.

\end{document}  % required; doc ends here

These lines are "comments" too, since they are past the end of the
document and thus will be ignored by LaTeX:

Suppose the name of this file is u.tex.  Then process it by typing

pdflatex u.tex

into a command window.  This produces the file u.pdf.  View it using
Acroread, xpdf or any other PDF viewer.

