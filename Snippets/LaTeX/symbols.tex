%%% symbols.tex version 6.0
%%% Original version written by David Carlisle, 1994/10/02.
%%% Massively changed and currently maintained by Scott Pakin <pakin@uiuc.edu>.
%%%
%%% To build this document:
%%%   latex symbols
%%%   makeindex -s gind.ist symbols
%%%   latex symbols
%%%   latex symbols
%%%

\NeedsTeXFormat{LaTeX2e}

\documentclass{article}
\usepackage{array}
\usepackage{longtable}
\usepackage{textcomp}
\usepackage{latexsym}
\usepackage{varioref}
\usepackage{xspace}
\usepackage{makeidx}
\usepackage{verbatim}


\newcommand{\doctitle}{Comprehensive \LaTeX\ Symbol List}  % Reusable
\title{The \doctitle}
\author{Scott Pakin \texttt{<pakin@uiuc.edu>}\thanks{The original
version of this document was written by David Carlisle, with several
additional tables provided by Alexander Holt.  See
Section~\vref{about-doc} for more information about who did what.}}
\date{July 2, 2001}

\makeindex

%%%
%%% TO-DO LIST
%%%   * Proofread, especially looking for symbols defined by more
%%%     than one symbol set or symbols that should be in a table
%%%     but aren't.
%%%   * Figure out how to make this file play nice with hyperref.
%%%   * See if there's anything good we can borrow from [Dow00].
%%%   * Add more symbol tables.  (Did we miss any common, standard, or
%%%     useful ones?)
%%%   * Further index symbols by _description_ (e.g., "perpendicular"
%%%     for "\perp").  This would be really useful, but extremely
%%%     time-consuming to do.  Note that Adobe's Web site has a list
%%%     of the names of all the Zapf Dingbats characters.  Unfortunately,
%%%     these names can be rather long, like "notched upper right-shadowed
%%%     white rightwards arrow" for \ding{241}.
%%%   * Find some way to associate each package with a flag indicating
%%%     whether the corresponding fonts are in bitmapped or vector
%%%     format.
%%%

% Indexed logical styles
\newcommand{\pkgname}[1]{%
  \textsf{#1}\index{#1=\textsf{#1}}\index{packages>\textsf{#1}}}
\newcommand{\optname}[1]{%
  \textsf{#1}\index{#1=\textsf{#1}}\index{options>\textsf{#1}}}

% Indexed common words and phrases
\newcommand{\latex}{\LaTeX\index{LaTeX=\LaTeX}\xspace}
\newcommand{\latexE}{\LaTeXe\index{LaTeX2e=\LaTeXe}\xspace}
\newcommand{\metafont}{\MF\index{Metafont=\MF}\xspace}
\newcommand{\TeXbook}{%
  The \TeX{}book\index{TeXbook, The=\TeX{}book, The}~\cite{Knuth:ct-a}\xspace}

% Index "X Y" and "Y, X".
\newcommand{\idxboth}[2]{\mbox{}\index{#1 #2}\index{#2>#1}}

% Index TeXbook symbols.
\newcommand{\idxTBsyms}{%
  \index{symbols>TeXbook=\TeX{}book}%
  \index{TeXbook, The=\TeX{}book, The>symbols from}%
}



%%%%%%%%%%%%%%%%%%%%%%%%%%%%%%%%%%%%%%%%%%%%%%%%%%%%%%%%%%%%%%%%%%%%%%%%%%
% There are a number of symbols (e.g., \Square) that are defined by      %
% multiple packages.  In order to typeset all the variants in this       %
% document, we have to give glyph a unique name.  To do that, we define  %
% \savesymbol{XXX}, which renames a symbol from \XXX to \origXXX, and    %
% \restoresymbols{yyy}{XXX}, which renames \origXXX back to \XXX and     %
% defines a new command, \yyyXXX, which corresponds to the most recently %
% loaded version of \XXX.                                                %
%                                                                        %

% Save a symbol that we know is going to get redefined.
\def\savesymbol#1{%
  \expandafter\let\expandafter\origsym\expandafter=\csname#1\endcsname
  \expandafter\let\csname orig#1\endcsname=\origsym
  \expandafter\let\csname#1\endcsname=\relax
}

% Restore a previously saved symbol, and rename the current one.
\def\restoresymbol#1#2{%
  \expandafter\let\expandafter\newsym\expandafter=\csname#2\endcsname
  \expandafter\global\expandafter\let\csname#1#2\endcsname=\newsym
  \expandafter\let\expandafter\origsym\expandafter=\csname orig#2\endcsname
  \expandafter\global\expandafter\let\csname#2\endcsname=\origsym
}

%                                                                        %
%%%%%%%%%%%%%%%%%%%%%%%%%%%%%%%%%%%%%%%%%%%%%%%%%%%%%%%%%%%%%%%%%%%%%%%%%%


% Each of the packages used by this document is loaded conditionally.
% However, it might be nice to know if we have a complete set.  So we
% define \ifcomplete which starts true, but gets set to false if any
% package is missing.
\newif\ifcomplete
\completetrue


% \IfStyFileExists* is just like \IfFileExists, except that it appends
% ".sty" to its first argument.  \IfStyFileExists is the same as
% \IfStyFileExists*, but it additionally adds its first argument to a list
% (\missingpkgs) and marks the document as incomplete (with
% \completefalse) if the .sty file doesn't exist.
\makeatletter
\newcommand{\missingpkgs}{}
\newcommand{\foundpkgs}{}
\newcommand{\if@sty@file@exists@star}[3]{\IfFileExists{#1.sty}{#2}{#3}}
\newcommand{\if@sty@file@exists}[3]{%
  \IfFileExists{#1.sty}%
               {#2\@cons\foundpkgs{{#1}}}%
               {#3\completefalse\@cons\missingpkgs{{#1}}}
}
\newcommand{\IfStyFileExists}{%
  \@ifstar{\if@sty@file@exists@star}{\if@sty@file@exists}
}
\makeatother

% We get a few packages for free.
\makeatletter
\@cons\foundpkgs{{textcomp}}
\@cons\foundpkgs{{latexsym}}
\makeatother
\newcommand{\TC}{\pkgname{textcomp}}

% The following was taken from amsmath.sty.
\makeatletter
  \newcommand{\AmSfont}{%
    \usefont{OMS}{cmsy}{\if\@xp\@car\f@series\@nil bb\else m\fi}{n}}
\makeatother
\providecommand{\AmS}{{\protect\AmSfont
  A\kern-.1667em\lower.5ex\hbox{M}\kern-.125emS}}

\newif\ifAMS
\newcommand\AMS{\AmS\index{AMS=\AmS}}
\IfStyFileExists{amssymb}
  {\AMStrue
   \savesymbol{angle} \savesymbol{rightleftharpoons}
   \usepackage{amssymb}
   \restoresymbol{AMS}{angle} \restoresymbol{AMS}{rightleftharpoons}
  }
  {}

\newif\ifST
\newcommand\ST{\pkgname{stmaryrd}}
\IfStyFileExists{stmaryrd}
  {\STtrue
   \savesymbol{lightning}
   \savesymbol{bigtriangleup} \savesymbol{bigtriangledown}
   \usepackage{stmaryrd}
   \restoresymbol{ST}{lightning}
   \restoresymbol{ST}{bigtriangleup} \restoresymbol{ST}{bigtriangledown}
  }
  {}

\newif\ifEU
\IfStyFileExists{euscript}
  {\EUtrue\usepackage[mathcal]{euscript}}
  {\let\CMcal\mathcal}

\newif\ifWASY
\newcommand\WASY{\pkgname{wasysym}}
\IfStyFileExists{wasysym}
  {\WASYtrue
   \savesymbol{lightning}
   \usepackage{wasysym}
   \restoresymbol{WASY}{lightning}
  }
  {}

\newif\ifPI
\newcommand\PI{\pkgname{pifont}}
\IfStyFileExists{pifont}
  {\PItrue\usepackage{pifont}}
  {}

\newif\ifMARV
\newcommand\MARV{\pkgname{marvosym}}
\IfStyFileExists{marvosym}
  {\MARVtrue
   \savesymbol{Rightarrow}
   \usepackage{marvosym}[2000/05/01]  % Major rewrite at this version.
   \restoresymbol{marv}{Rightarrow}
  }
  {}

\newif\ifMAN
\newcommand\MAN{\pkgname{manfnt}}
\IfStyFileExists{manfnt}
  {\MANtrue\usepackage{manfnt}}
  {}

\newif\ifDING
\newcommand\DING{\pkgname{bbding}}
\IfStyFileExists{bbding}
  {\DINGtrue
   \savesymbol{Cross} \savesymbol{Square}
   \usepackage{bbding}
   \restoresymbol{ding}{Cross} \restoresymbol{ding}{Square}
  }
  {}

\newif\ifIFS
\newcommand\IFS{\pkgname{ifsym}}
\IfStyFileExists{ifsym}
  {\IFStrue
   \savesymbol{Letter} \savesymbol{Square} \savesymbol{Cross} \savesymbol{Sun}
   \savesymbol{TriangleUp} \savesymbol{TriangleDown} \savesymbol{Circle}
   \usepackage[alpine,clock,electronic,geometry,misc,weather]{ifsym}
   \restoresymbol{ifs}{Letter} \restoresymbol{ifs}{Square}
   \restoresymbol{ifs}{Cross} \restoresymbol{ifs}{Sun}
   \restoresymbol{ifs}{TriangleUp} \restoresymbol{ifs}{TriangleDown}
   \restoresymbol{ifs}{Circle}
  }
  {}

\newif\ifTIPA
\newcommand\TIPA{\pkgname{tipa}}
\IfStyFileExists{tipa}
  {\TIPAtrue\usepackage[safe]{tipa}}
  {}

% We use the *-form of \IfStyFileExists, because the package is named
% "wsuipa", while the .sty file is named "ipa.sty".
\makeatletter
\newif\ifWIPA
\newcommand\WIPA{\pkgname{wsuipa}}
\IfStyFileExists*{ipa}
  {\@cons\foundpkgs{{wsuipa}}
   \WIPAtrue
   \savesymbol{baro} \savesymbol{eth} \savesymbol{openo} \savesymbol{thorn}
   \usepackage{ipa}
   \restoresymbol{WSU}{baro}  \restoresymbol{WSU}{eth}
   \restoresymbol{WSU}{openo} \restoresymbol{WSU}{thorn}
  }
  {\completefalse\@cons\missingpkgs{{wsuipa}}}
\makeatother

\newif\ifULSY
\newcommand\ULSY{\pkgname{ulsy}}
\IfStyFileExists{ulsy}
  {\ULSYtrue\usepackage{ulsy}}
  {}

\newif\ifASP
\newcommand\ASP{\pkgname{ar}}
\IfStyFileExists{ar}
  {\ASPtrue\usepackage{ar}}
  {}

% pxfonts relies on txfonts (I think), so either package can be loaded.
\newif\ifTX
\newcommand\TX{\pkgname{txfonts}}
\newcommand\PX{\pkgname{pxfonts}}
\newcommand\TXPX{\pkgname{txfonts}/\pkgname{pxfonts}}
\makeatletter
\IfStyFileExists{txfonts}
  {\TXtrue\usepackage{txfonts}
   % Restore the default fonts.
   \renewcommand\rmdefault{cmr}
   \renewcommand\sfdefault{cmss}
   \renewcommand\ttdefault{cmtt}
   \ifAMS
     \DeclareMathAlphabet\mathfrak{U}{euf}{m}{n}
   \fi
   % Are \textcent, \textsterling, \L, \l, and \r the only symbols that
   % get screwed up?
   \let\origtextcent=\textcent
   \gdef\textcent{{\fontencoding{TS1}\selectfont\origtextcent}}
   \let\origtextsterling=\textsterling
   \gdef\textsterling{{\fontencoding{TS1}\selectfont\origtextsterling}}
   \DeclareTextCommand{\L}{OT1}
     {\leavevmode\setbox\z@\hbox{L}\hb@xt@\wd\z@{\hss\@xxxii L}}
   \DeclareTextCommand{\l}{OT1}
     {{\@xxxii l}}
   \DeclareTextAccent{\r}{OT1}{23}
  }
  {}
\makeatother

\newif\ifFC
\newcommand\FC{\pkgname{fc}}
\IfStyFileExists{fclfont}
  {\FCtrue
   \let\origlbrace=\{
   \let\origrbrace=\}
   \let\origbar=\|
   \let\origdollar=\$
   \let\origspace=\_
   \let\origS=\S
   \let\origpounds=\pounds
   \input{t4enc.def}
   \global\let\{=\origlbrace
   \global\let\}=\origrbrace
   \global\let\|=\origbar
   \global\let\$=\origdollar
   \global\let\_=\origspace
   \global\let\S=\origS
   \global\let\pounds=\origpounds
  }
  {}

\newif\ifASCII
\newcommand\ASCII{\pkgname{ascii}}
\IfStyFileExists{ascii}
  {\ASCIItrue\usepackage{ascii}}
  {}

\newif\ifARK                           % ark10 and dingbat fonts
\newcommand\ARK{\pkgname{dingbat}}
\IfStyFileExists{dingbat}
  {\ARKtrue
   \savesymbol{checkmark}
   \usepackage{dingbat}
   \restoresymbol{ARK}{checkmark}
  }
  {}

% yfonts re-encodes \aa and \AA as LY, so we have to re-re-encode them
% as OT1.
\IfStyFileExists{yfonts}
  {\usepackage{yfonts}
   \DeclareTextCommand{\aa}{OT1}{{\accent23a}}
   \DeclareTextCommand{\AA}{OT1}{{\accent23A}}}
  {}

% If we have mflogo.sty, use it.  Otherwise, define "\MF" the "boring" way.
\IfStyFileExists*{mflogo}
  {\usepackage{mflogo}}
  {\newcommand{\MF}{Metafont}}

% If we have booktabs.sty, use it.  Otherwise, define all its line types
% in terms of \hline and \cline.
\IfStyFileExists*{booktabs}
  {\usepackage{booktabs}}
  {\newcommand{\toprule}{\hline}
   \newcommand{\midrule}{\hline}
   \newcommand{\bottomrule}{\hline}
   \def\cmidrule(##1)##2{\cline{##2}}
  }

% If we have url.sty, use it.  Otherwise, define \url as \texttt.
\IfStyFileExists*{url}
  {\usepackage{url}
   \def\UrlBreaks{}
   \def\UrlBigBreaks{\do/}}
  {\newcommand{\url}[1]{\texttt{##1}}}

% If we have geometry.sty, use it.  Otherwise, a lot of tables are going
% to stick out into the margin.
\IfStyFileExists*{geometry}
  {\usepackage{geometry}}
  {}

% If we have multicol.sty, load it.
\newif\ifhavemulticol
\IfStyFileExists*{multicol}
  {\havemulticoltrue\usepackage{multicol}}
  {}

% amsmath causes all sorts of conflicts with the rest of this document.
% As a workaround, we extract the minimum we need from amsmath, then load
% only amsopn.  (Doing this earlier *also* causes conflicts. :-( )
\ifAMS
  \makeatletter
    \def\arrowfill@#1#2#3#4{%
      $\m@th\thickmuskip0mu\medmuskip\thickmuskip\thinmuskip\thickmuskip
       \relax#4#1\mkern-7mu%
       \cleaders\hbox{$#4\mkern-2mu#2\mkern-2mu$}\hfill
       \mkern-7mu#3$%
    }
    \def\leftarrowfill@{\arrowfill@\leftarrow\relbar\relbar}
    \def\rightarrowfill@{\arrowfill@\relbar\relbar\rightarrow}
    \def\leftrightarrowfill@{\arrowfill@\leftarrow\relbar\rightarrow}
    \let\@@overline=\overline
    \def\varinjlim{%
      \mathop{\mathpalette\varlim@{\rightarrowfill@\textstyle}}\nmlimits@
    }
    \def\overarrow@#1#2#3{\vbox{\ialign{##\crcr#1#2\crcr
     \noalign{\nointerlineskip}$\m@th\hfil#2#3\hfil$\crcr}}}
    \renewcommand{\overrightarrow}{%
      \mathpalette{\overarrow@\rightarrowfill@}}
    \renewcommand{\overleftarrow}{%
      \mathpalette{\overarrow@\leftarrowfill@}}
    \newcommand{\overleftrightarrow}{%
      \mathpalette{\overarrow@\leftrightarrowfill@}}
    \def\underarrow@#1#2#3{%
     \vtop{\ialign{##\crcr$\m@th\hfil#2#3\hfil$\crcr
     \noalign{\nointerlineskip\kern1.3\ex@}#1#2\crcr}}}
    \newcommand{\underrightarrow}{%
      \mathpalette{\underarrow@\rightarrowfill@}}
    \newcommand{\underleftarrow}{%
      \mathpalette{\underarrow@\leftarrowfill@}}
    \newcommand{\underleftrightarrow}{%
      \mathpalette{\underarrow@\leftrightarrowfill@}}
  \makeatother
  \usepackage{amsopn}
\fi

% If we have the accents package, use it (for an example in the section
% on constructing new symbols).
\newif\ifACCENTS
\IfStyFileExists{accents}
  {\ACCENTStrue
   \savesymbol{undertilde}
   \usepackage{accents}
   \restoresymbol{ACCENTS}{undertilde}
  }
  {}


%%%%%%%%%%%%%%%%%%%%%%%%%%%%%%%%%%%%%%%%%%%%%%%%%%%%%%%%%%%%%%%%%%%%%%%%%%
% Because most (La)TeX builds are limited to 16 math alphabets, we       %
% define our own _text_ commands below instead of doing a \usepackage,   %
% because the latter would invoke a \DeclareMathAlphabet.                %
%                                                                        %

\IfStyFileExists{mathrsfs}
  {\newcommand{\mathscr}[1]{\mbox{\usefont{U}{rsfs}{m}{n} ##1}}}
  {}

\IfStyFileExists{zapfchan}
  {\newcommand{\mathpzc}[1]{\mbox{\usefont{OT1}{pzc}{m}{it} ##1}}}
  {}

\IfStyFileExists{bbold}
  {\newcommand{\BBmathbb}[1]{\mbox{\usefont{U}{bbold}{m}{n} ##1}}}
  {}

\IfStyFileExists{dsfont}
  {\newcommand{\mathds}[1]{\mbox{\usefont{U}{dsrom}{m}{n} ##1}}
   \newcommand{\mathdsss}[1]{\mbox{\usefont{U}{dsss}{m}{n} ##1}}}
  {}

\IfStyFileExists{bbm}
  {\newcommand{\mathbbm}[1]{\mbox{\usefont{U}{bbm}{m}{n} ##1}}
   \newcommand{\mathbbmss}[1]{\mbox{\usefont{U}{bbmss}{m}{n} ##1}}
   \newcommand{\mathbbmtt}[1]{\mbox{\usefont{U}{bbmtt}{m}{n} ##1}}}
  {}

%                                                                        %
%                                                                        %
%%%%%%%%%%%%%%%%%%%%%%%%%%%%%%%%%%%%%%%%%%%%%%%%%%%%%%%%%%%%%%%%%%%%%%%%%%


% Resolve the stmaryrd/wasysym \lightning conflict by defining \lightning
% to use stmaryrd in math mode and wasysym in text mode.
\DeclareRobustCommand{\lightning}{\ifmmode\STlightning\else\WASYlightning\fi}

% Index a symbol, which may or may not begin with a backslash.
% (Is there a better way to do this?)
\begingroup
 \catcode`\|=0
 \catcode`\\=12
 |gdef|sanitize#1#2!!!{%
   |ifx#1\%
     #2%
   |else%
     #1#2%
   |fi%
}
|endgroup
\newcommand{\indexcommand}[1]{%
  \edef\sanitized{\expandafter\sanitize\string#1!!!}%
  \expandafter\index\expandafter{\sanitized=\string\verb+\string#1+}%
}

% Enable the use of makeindex's nicer-looking gind.ist style.
% I swiped the following from doc.dtx.
\makeatletter
\newif\ifscan@allowed
\def\efill{\hfill\nopagebreak}%
\def\dotfill{\leaders\hbox to.6em{\hss .\hss}\hskip\z@ plus 1fill}%
\def\dotfil{\leaders\hbox to.6em{\hss .\hss}\hfil}%
\def\pfill{\unskip~\dotfill\penalty500\strut\nobreak
           \dotfil~\ignorespaces}%
\makeatother

% If we have the multicol package, typeset the index in three columns instead
% of the usual two.
\ifhavemulticol
  \makeatletter
  \renewenvironment{theindex}{%
    \clearpage
    \setlength{\columnsep}{1em}%
    \begin{multicols}{3}[%
      \section*{\indexname}
      If you're having trouble locating a symbol, try looking under ``T''
      for ``\texttt{\string\text}$\ldots$''.  Many text-mode commands begin
      with that prefix.
    ]%
    \let\item\@idxitem
  }{%
    \end{multicols}%
  }
  \makeatother
\fi

% Define a counter to keep track of how many symbols are listed.
% Output this counter to the log file at the end of each run.
% Define \prevtotalsymbols to be the total number of symbols from
% the previous run.
\newcounter{totalsymbols}
\newcommand{\incsyms}{\addtocounter{totalsymbols}{1}}
\makeatletter
\AtEndDocument{%
  \typeout{Number of symbols documented: \thetotalsymbols}
  \immediate\write\@auxout{%
    \noexpand\gdef\noexpand\prevtotalsymbols{\thetotalsymbols}}
}
\makeatother

% Define \prevtotalsymbols as "??" if this is our first run.
\expandafter\ifx\csname prevtotalsymbols\endcsname\relax
\def\prevtotalsymbols{\textbf{??}}
\fi


% Symbol+verbatim for various types of symbols
\def\Jf#1#2{\incsyms\indexcommand{#1}{\fontencoding{T4}\selectfont#1#2} &
  \ttfamily\string#1\string{#2\string}}
\makeatletter
  \def\K@opt@arg[#1]#2{\incsyms\indexcommand{#2}#1 &\ttfamily\string#2}
  \def\K@no@opt@arg#1{\incsyms\indexcommand{#1}#1 &\ttfamily\string#1}
  \def\K{\@ifnextchar[{\K@opt@arg}{\K@no@opt@arg}}
\makeatother
\def\Ka#1{\incsyms\indexcommand{#1}{\ascii#1} &\ttfamily\string#1}
\def\Ks#1{\incsyms\indexcommand{#1}{\fontencoding{T1}\selectfont#1} &\ttfamily\string#1$^*$}
\def\Kt#1{\incsyms\indexcommand{#1}{\fontencoding{T1}\selectfont#1} &\ttfamily\string#1}
\def\N#1{\incsyms\indexcommand{#1}$#1$ & $\Big#1$ &\ttfamily\string#1}
\def\Q#1{\incsyms\indexcommand{#1}#1{A}#1{a} &
         \ttfamily\string#1\string{A\string}\string#1\string{a\string}}
\def\Qt#1{\incsyms\indexcommand{#1}{\fontencoding{T1}\selectfont#1{A}#1{a}} &
          \ttfamily\string#1\string{A\string}\string#1\string{a\string}}
\def\Qf#1{\incsyms\indexcommand{#1}{\fontencoding{T4}\selectfont#1{A}#1{a}} &
          \ttfamily\string#1\string{A\string}\string#1\string{a\string}}
\makeatletter
  % We use \displaystyle so that variable-sized symbols will be big.
  \def\R@opt@arg[#1]#2{\incsyms\indexcommand{#2}$#1$ & $\displaystyle#1$ &\ttfamily\string#2}
  \def\R@no@opt@arg#1{\incsyms\indexcommand{#1}$#1$ & $\displaystyle#1$ &\ttfamily\string#1}
  \def\R{\@ifnextchar[{\R@opt@arg}{\R@no@opt@arg}}
\makeatother
\def\Tp#1{\incsyms\indexcommand{\ding}\ding{#1} &\ttfamily\string\ding\string{#1\string}}
\newcommand{\V}[2][]{\incsyms#1 & \indexcommand{#2}#2 &\ttfamily\string#2}
\def\W#1#2{\incsyms\indexcommand{#1}$#1{#2}$ &\ttfamily\string#1\string{#2\string}}
\def\Ww#1#2#3{\incsyms\indexcommand{#2}$#1{#3}$ &\ttfamily\string#2\string{#3\string}}
\makeatletter
  \def\X@opt@arg[#1]#2{\incsyms\indexcommand{#2}$#1$ &\ttfamily\string#2}
  \def\X@no@opt@arg#1{\incsyms\indexcommand{#1}$#1$ &\ttfamily\string#1}
  \def\X{\@ifnextchar[{\X@opt@arg}{\X@no@opt@arg}}
\makeatother
\def\Y#1{\incsyms\indexcommand{#1}$\big#1$ & $\Bigg#1$ &\ttfamily\string#1}
\def\Z#1{\incsyms\indexcommand{#1}\ttfamily\string#1}

% Display and index a command, but not its symbol.
\def\cmd#1{\texttt{\string#1}\indexcommand{#1}}


% Redefine the LaTeX commands that are replaced by textcomp.
% This was swiped right out of ltoutenc.dtx, but with "\text..."
% changed to "\ltext...".
\DeclareTextCommandDefault{\ltextcopyright}{\textcircled{c}}
\DeclareTextCommandDefault{\ltextregistered}{\textcircled{\scshape r}}
\DeclareTextCommandDefault{\ltexttrademark}{\textsuperscript{TM}}
\DeclareTextCommandDefault{\ltextordfeminine}{\textsuperscript{a}}
\DeclareTextCommandDefault{\ltextordmasculine}{\textsuperscript{o}}


% Needed by the References section.  This was copy&pasted from ltlogos.dtx.
\makeatletter
\DeclareRobustCommand{\LaT}{L\kern-.36em%
        {\sbox\z@ T%
         \vbox to\ht\z@{\hbox{\check@mathfonts
                              \fontsize\sf@size\z@
                              \math@fontsfalse\selectfont
                              A}%
                        \vss}%
        }%
        \kern-.15em T%
}
\makeatother

% Display a metavariable.
\newcommand{\meta}[1]{$\langle$\textit{#1}$\rangle$}

% Many tables have notes beneath them.  Define an environment in which to
% display such a note, with an optional, superscripted math symbol
% preceding it.
\newenvironment{tablenote}[1][]{
  \makebox[1em]{\ensuremath{^{#1}}}%
  \begin{minipage}[t]{0.75\textwidth}%
  \setlength{\parskip}{2ex}
}{%
  \end{minipage}%
}

% Define a couple of messages we reuse repeatedly.
\newcommand{\twosymbolmessage}{%
  \begin{tablenote}
    Where two symbols are present, the left one is the ``faked'' symbol
    that \latexE{} provides by default, and the right one is the ``true''
    symbol that \TC\ makes available.
  \end{tablenote}
}

\newcommand{\notpredefinedmessage}{%
  \begin{tablenote}[*]
    Not predefined in \latexE.  Use one of the packages
    \pkgname{latexsym}, \pkgname{amsfonts}, \pkgname{amssymb},
    \pkgname{txfonts}, \pkgname{pxfonts}, or \pkgname{wasysym}.
  \end{tablenote}
}


% Define an environment in which to write a single table of symbols.  The
% environment looks a lot like a table, but it doesn't float, and it gets
% an entry in the table of contents (as a subsubsection that looks like a
% subsection), as opposed to the list of tables.
%
% The first argument is a conditional.  The table will appear only if
% the value of the conditional is true.  The second argument is the
% table's caption.
\makeatletter
\def\fnum@table{\textsc{\tablename}~\thetable}
\newenvironment{symtable}[2][true]{%
  \expandafter\global\expandafter\let%
    \expandafter\ifshowsymtable\csname if#1\endcsname
  \ifshowsymtable
    \noindent%
    \begin{minipage}[t]{\linewidth}    % Prevent page breaks
    \begin{center}
    \addtocounter{table}{1}%
    \protected@edef\@currentlabel{\thetable}%
    \addcontentsline{toc}{subsubsection}{%
      \protect\numberline{\tablename~\thetable:}{#2}}%
    \@makecaption{\fnum@table}{#2}\medskip
  \else
    % The following was taken verbatim from verbatim.sty.
    \let\do\@makeother\dospecials\catcode`\^^M\active
    \let\verbatim@startline\relax
    \let\verbatim@addtoline\@gobble
    \let\verbatim@processline\relax
    \let\verbatim@finish\relax
    \verbatim@
  \fi
}{%
  \ifshowsymtable
    \end{center}
    \end{minipage}
    \vskip 8ex minus 2ex
  \fi
}
\makeatother

% Same as the above, but allows page breaks.
\makeatletter
\newenvironment{longsymtable}[2][true]{%
  \expandafter\global\expandafter\let%
    \expandafter\ifshowsymtable\csname if#1\endcsname
  \ifshowsymtable
    \begin{center}%
    \addtocounter{table}{1}%
    \protected@edef\@currentlabel{\thetable}%
    \addcontentsline{toc}{subsubsection}{%
      \protect\numberline{\tablename~\thetable:}{#2}}%
    \@makecaption{\fnum@table}{#2}%
    \def\lt@indexed{}%
  \else
    % The following was taken verbatim from verbatim.sty.
    \let\do\@makeother\dospecials\catcode`\^^M\active
    \let\verbatim@startline\relax
    \let\verbatim@addtoline\@gobble
    \let\verbatim@processline\relax
    \let\verbatim@finish\relax
    \verbatim@
  \fi
}{%
  \ifshowsymtable
    \let\@elt=\index\lt@indexed  % Close our index ranges.
    \end{center}
    \addtocounter{table}{-1}     % Make up for longtable's counter increment.
    \vskip 8ex minus 2ex
  \fi
}
\makeatother

% Define \index-like commands for use with longsymtable that
% automatically apply to the entire table, not just the start of it.
\makeatletter
\newcommand{\ltindex}[1]{%
  \index{#1|(}%
  \@cons{\lt@indexed}{{#1|)}}%
}
\newcommand{\ltidxboth}[2]{\mbox{}\ltindex{#1 #2}\ltindex{#2>#1}}
\makeatother


% Define a table environment that's similar to symtable, except that it
% floats and it doesn't write an entry into the Table of Contents.  This
% is used for tables that contain something other than symbol lists.
\makeatletter
\newenvironment{nonsymtable}[1]{%
  \begin{table}[htbp]
  \centering
  \caption{#1}\medskip
}{%
  \end{table}
}
\makeatother

% Make sure we have enough room in the table of contents for
% the word "Table" at the beginning of each symtable entry.
\makeatletter
\settowidth{\@tempdimc}{Table~999:\hspace*{0.5em}}
\renewcommand*\l@subsubsection{\@dottedtocline{3}{1.5em}{\the\@tempdimc}}
\makeatother

% Paragraphs with tall symbols should get a little extra interline spacing.
\newenvironment{morespacing}[1]{\advance\baselineskip by #1\relax}{\par}

% Use Donald Arseneau's improved float parameters.
\renewcommand{\topfraction}{.85}
\renewcommand{\bottomfraction}{.7}
\renewcommand{\textfraction}{.15}
\renewcommand{\floatpagefraction}{.66}
\renewcommand{\dbltopfraction}{.66}
\renewcommand{\dblfloatpagefraction}{.66}
\setcounter{topnumber}{9}
\setcounter{bottomnumber}{9}
\setcounter{totalnumber}{20}
\setcounter{dbltopnumber}{9}

%%%%%%%%%%%%%%%%%%%%%%%%%%%%%%%%%%%%%%%%%%%%%%%%%%%%%%%%%%%%%%%%%%%%%%%%%%%

\begin{document}
\sloppy
\maketitle

\begin{abstract}
  This document lists \prevtotalsymbols{} symbols and the corresponding
  \latex{} commands that produce them.  Some of these symbols are
  guaranteed to be available in every \latexE{} system; others require
  fonts and packages that may not accompany a given distribution and
  that therefore need to be installed.  All of the fonts and packages
  used to prepare this document---as well as this document itself---are
  freely available from the Comprehensive\index{CTAN} \TeX{} Archive
  Network (\url{http://www.ctan.org}).
\end{abstract}

\tableofcontents

% Now that we've output the table of contents, let's make \section start a
% new page.  I toyed with the idea of changing the documentclass from
% article to report, but I like having the abstract on the same page as
% the title and the start of the table of contents; I want the tables
% numbered consecutively throughout the document; and I like the smaller,
% less gaudy section headings the article class offers.  In short, article
% seems a better fit than report.
\makeatletter
\let\origsection=\section
\renewcommand\section{\@startsection {section}{1}{\z@}%
                                     {-3.5ex \@plus -1ex \@minus -.2ex}%
                                     {2.3ex \@plus.2ex}%
                                     {\clearpage\normalfont\Large\bfseries}}
\makeatother


% Define an integral containing a dash or a double-dash.
\def\Xint#1{\mathchoice
   {\XXint\displaystyle\textstyle{#1}}%
   {\XXint\textstyle\scriptstyle{#1}}%
   {\XXint\scriptstyle\scriptscriptstyle{#1}}%
   {\XXint\scriptscriptstyle\scriptscriptstyle{#1}}%
   \!\int}
\def\XXint#1#2#3{{\setbox0=\hbox{$#1{#2#3}{\int}$}
     \vcenter{\hbox{$#2#3$}}\kern-.5\wd0}}
\def\ddashint{\Xint=}
\def\dashint{\Xint-}


% Many symbols are merely alphanumerics typeset with a math alphabet.
% Guide the user from the most common of these to the Math Alphabets
% table.
%
% QUESTION: How standard are the following?
%    * Bernoulli (script B)
%    * domain (script D)
%    * expected value (script E)
%    * Hamiltonian (script H)
%    * imaginary line (script I)
%    * M matrix (script M)
%    * Mellin transform (script M)
%    * null space (script N)
%    * order of (script o)
%    * real line (script R)
%    * Schwartz class (script S)
%
\index{Fourier transform|see{alphabets, math}}
\index{Hilbert space|see{alphabets, math}}
\index{Lagrangian|see{alphabets, math}}
\index{Laplace transform|see{alphabets, math}}
\index{cardinality|see{\texttt{\string\aleph}}}
\index{complex numbers|see{alphabets, math}}
\index{imaginary numbers|see{alphabets, math}}    % or \Im
\index{integers|see{alphabets, math}}
\index{natural numbers|see{alphabets, math}}
\index{number sets|see{alphabets, math}}
\index{prime numbers|see{alphabets, math}}
\index{quaternions|see{alphabets, math}}
\index{rational numbers|see{alphabets, math}}
\index{real numbers|see{alphabets, math}}         % or \Re
\index{unity|see{alphabets, math}}

% Provide "see also"s for every accent whose name I happen to know.
\index{acute|see{accents}}
\index{breve|see{accents}}
\index{caret|see{accents}}
\index{cedilla|see{accents}}
\index{circumflex|see{accents}}
\index{diaeresis=di\ae{}resis|see{accents}}
\index{grave|see{accents}}
\index{hacek=h\'{a}\v{c}ek|see{accents}}
\index{Hungarian umlaut|see{accents}}
\index{macron|see{accents}}
\index{ogonek|see{accents}}
\index{ring|see{accents}}
\index{tilde|see{accents}}
\index{umlaut|see{accents}}

% Provide a few other useful "see also"s.
\index{Comprehensive TeX Archive Network=Comprehensive \TeX{} Archive Network|see{CTAN}}
\index{script letters|see{alphabets, math}}
\index{letters|see{alphabets}}
\index{numbers|see{digits}}
\index{numerals|see{digits}}
\index{degrees|see{\texttt{\string\textdegree}}}
\index{registered trademark|see{\texttt{\string\textregistered}}}
\index{Cedi|see{\texttt{\string\textcolonmonetary}}}
\index{iff=\texttt{\string\iff}|see{\texttt{\string\Longleftrightarrow}}}
\index{to=\texttt{\string\to}|see{\texttt{\string\rightarrow}}}
\ifAMS
  \index{implies=\texttt{\string\implies}|see{\texttt{\string\Longrightarrow}}}
  \index{impliedby=\texttt{\string\impliedby}|see{\texttt{\string\Longleftarrow}}}
\fi
\ifTIPA
  \index{symbols>dictionary|see{symbols, phonetic}}
  \index{dictionary symbols|see{phonetic symbols}}
  \index{pronunciation symbols|see{phonetic symbols}}
\fi    % TIPA test
\index{abzuglich=abz\"uglich|see{\texttt{\string\textdiscount}}}
\index{diacritics|see{accents}}


% The following were generated automatically from txfonts.sty.
\ifTX
\index{circledplus=\texttt{\string\circledplus}|see{\texttt{\string\oplus}}}
\index{circledminus=\texttt{\string\circledminus}|see{\texttt{\string\ominus}}}
\index{circledtimes=\texttt{\string\circledtimes}|see{\texttt{\string\otimes}}}
\index{circledslash=\texttt{\string\circledslash}|see{\texttt{\string\oslash}}}
\index{circleddot=\texttt{\string\circleddot}|see{\texttt{\string\odot}}}
\index{le=\texttt{\string\le}|see{\texttt{\string\leq}}}
\index{ge=\texttt{\string\ge}|see{\texttt{\string\geq}}}
\index{gets=\texttt{\string\gets}|see{\texttt{\string\leftarrow}}}
\index{to=\texttt{\string\to}|see{\texttt{\string\rightarrow}}}
\index{owns=\texttt{\string\owns}|see{\texttt{\string\ni}}}
\index{lnot=\texttt{\string\lnot}|see{\texttt{\string\neg}}}
\index{land=\texttt{\string\land}|see{\texttt{\string\wedge}}}
\index{lor=\texttt{\string\lor}|see{\texttt{\string\vee}}}
\index{restriction=\texttt{\string\restriction}|see{\texttt{\string\upharpoonright}}}
\index{Doteq=\texttt{\string\Doteq}|see{\texttt{\string\doteqdot}}}
\index{doublecup=\texttt{\string\doublecup}|see{\texttt{\string\Cup}}}
\index{doublecap=\texttt{\string\doublecap}|see{\texttt{\string\Cap}}}
\index{llless=\texttt{\string\llless}|see{\texttt{\string\lll}}}
\index{gggtr=\texttt{\string\gggtr}|see{\texttt{\string\ggg}}}
%\index{Box=\texttt{\string\Box}|see{\texttt{\string\square}}}
\index{ne=\texttt{\string\ne}|see{\texttt{\string\neq}}}
\index{notowns=\texttt{\string\notowns}|see{\texttt{\string\notni}}}
\index{lrJoin=\texttt{\string\lrJoin}|see{\texttt{\string\Join}}}
%\index{bowtie=\texttt{\string\bowtie}|see{\texttt{\string\lrtimes}}}
\index{dasharrow=\texttt{\string\dasharrow}|see{\texttt{\string\dashrightarrow}}}
\index{circledotright=\texttt{\string\circledotright}|see{\texttt{\string\circleddotright}}}
\index{circledotleft=\texttt{\string\circledotleft}|see{\texttt{\string\circleddotleft}}}

\fi    % TX test


\section{Introduction}

Welcome to the \doctitle!  This document strives to be your primary
source of \latex{} symbol information: font samples, \latex{}
commands, packages, usage details, caveats---everything needed to put
thousands of different symbols at your disposal.  All of the fonts
covered herein meet the following criteria:

\begin{enumerate}
  \item They are freely available from the Comprehensive\index{CTAN}
  \TeX{} Archive Network (\url{http://www.ctan.org}).

  \item All of their symbols have \latexE{} names.  That is, a user
  should be able to access a symbol by name, not just by
  \cmd{\char}\meta{number}.
\end{enumerate}

\noindent
These are not particularly limiting criteria; the \doctitle{} contains
samples of \prevtotalsymbols{} symbols---quite a large number.  Some
of these symbols are guaranteed to be available in every \latexE{}
system; others require fonts and packages that may not accompany a
given distribution and that therefore need to be installed.  See
\url{http://www.tex.ac.uk/cgi-bin/texfaq2html?label=instpackages+wherefiles}
for help with installing new fonts and packages.


\subsection*{Document Usage}

Each section of this document contains a number of font tables.  Each
table shows a set of symbols, with the corresponding \latex{} command
to the right of each symbol.  A table's caption indicates what package
needs to be loaded in order to access that table's symbols.  For
example, the symbols in Table~\ref{old-style-nums}, ``\TC\ Old-Style
Numerals'', are made available by putting
``\cmd{\usepackage}\verb|{textcomp}|'' in your document's preamble.
``\AMS'' means to use one of the \AMS{} symbol packages, such as
\pkgname{amssymb}.  Notes below a table provide additional information
about some or all the symbols in that table.  

One note that appears a few times in this document, particularly in
Section~\ref{body-text-symbols}, indicates that certain symbols do not
exist in the OT1 font encoding (Donald\index{Knuth, Donald E.} Knuth's
original, 7-bit font encoding, which is the default font encoding for
\latex) and that you should use \pkgname{fontenc} to select a
different encoding, such as T1 (a common 8-bit font encoding).  That
means that you should put
``\cmd{\usepackage}\verb|[|\meta{encoding}\verb|]{fontenc}|'' in your
document's preamble, where \meta{encoding} is, e.g., \texttt{T1} or
\texttt{LY1}.  To limit the change in font encoding to the current
group, use
``\cmd{\fontencoding}\verb|{|\meta{encoding}\verb|}|\cmd{\selectfont}''.

Section~\ref{addl-info} contains some additional information about the
symbols in this document.  It shows which symbol names are not unique
across packages, gives examples of how to create new symbols out of
existing symbols, explains how symbols are spaced in math mode,
presents a \latex{} ASCII\index{ASCII} and Latin~1\index{Latin 1}
tables, and provides some information about this document itself.  The
\doctitle{} ends with an index of all the symbols in the document and
various additional useful terms.


\ifcomplete

\subsection*{Frequently Requested Symbols}

There are a number of symbols that are requested over and over again
on \texttt{comp.text.tex}\index{comp.text.tex=\texttt{comp.text.tex}}.
If you're looking for such a symbol, the following list will help you
find it quickly.

\newenvironment{symbolfaq}{%
  \ifhavemulticol
    \setlength{\columnsep}{3em}%
    \begin{multicols}{2}%
  \fi
  \setlength{\parskip}{1ex}%
  \newcommand{\faq}[2]{%
    \noindent##1\quad\dotfill\quad\makebox[1em][r]{##2}\par}%
}{%
  \ifhavemulticol
    \end{multicols}%
  \fi
}

\begin{symbolfaq}
  \faq{\textcopyright\ and \textregistered}
      {\pageref{text-predef}}
  \faq{\textvisiblespace, as in
       ``Spaces\textvisiblespace are\textvisiblespace significant.''}
      {\pageref{text-predef}}
  \faq{\'{\i}, \`{\i}, \={\i}, \^{\i}, etc.\ (versus \'i, \`i, \=i, and \^i)}
      {\pageref{text-accents}}
  \faq{\textcent}
      {\pageref{tc-currency}}
  \faq{\EUR}
      {\pageref{marv-currency}}
  \faq{\textperthousand}
      {\pageref{tc-misc}}
  \faq{$\oiint$}
      {\pageref{txpx-large}}
  \faq{$\therefore$}
      {\pageref{ams-rel}}
  \faq{$\coloneqq$ and $\Coloneqq$}
      {\pageref{txpx-rel}}
  \faq{\textdegree, as in ``180\textdegree'' or ``15\textcelsius''}
      {\pageref{tc-math-science}}
  \faq{\mathscr{L}, \mathscr{F}, etc.}
      {\pageref{alphabets}}
  \faq{\mathbbm{N}, \mathbbm{Z}, \mathbbm{R}, etc.}
      {\pageref{alphabets}}
  \faq{$\mathinner{\mkern1mu\raise1pt
        \vbox{\kern7pt\hbox{.}}\mkern2mu
        \raise4pt\hbox{.}\mkern2mu\raise7pt\hbox{.}\mkern1mu}$}
      {\pageref{revddots}}
  \faq{\diatop[{\diatop[\'|\=]}|a],
       \diatop[{\diatop[\`|\^]}|e], etc.
       (i.e., several accents per character)}
      {\pageref{multiple-accents}}
  \faq{$\dashint$}
      {\pageref{dashint}}
  \faq{$<$ and $>$ (instead of < and >)}
      {\pageref{upside-down}}
  \faq{\textasciitilde\ (or $\sim$)}
      {\pageref{tildes}}
\end{symbolfaq}

\fi    % ifcomplete


\section{Body-text symbols}
\label{body-text-symbols}
\index{symbols>body text}

This section lists symbols that are intended for use in running text,
such as punctuation marks, accents, ligatures, and currency symbols.

\bigskip

\begin{symtable}{\latexE{} Escapable ``Special'' Characters}
\index{special characters}\index{escapable characters}
\label{special-escapable}
\begin{tabular}{*6{ll@{\qquad}}ll}
\K\$              & \K\%              & \K\_           & \K\}      &
\K\&              & \K\#              & \K\{
\end{tabular}
\end{symtable}


\begin{symtable}{\latexE{} Commands Defined to Work in Both Math and Text Mode}
\label{math-text}
\begin{tabular}{*3{lll@{\qquad}}lll}
\V\$              & \V\_              & \V\ddag           & \V\{      \\
\V\P              & \V[\ltextcopyright]\copyright
                                      & \V\dots           & \V\}      \\
\V\S              & \V\dag            & \V\pounds         \\
\end{tabular}

\bigskip
\twosymbolmessage
\end{symtable}


\begin{symtable}{Predefined \latexE{} Text-Mode Commands}
\label{text-predef}
\begin{tabular}{*6l}
\V\textasciicircum     & \V\textless            \\
\V\textasciitilde      & \V[\ltextordfeminine]\textordfeminine    \\
\V\textasteriskcentered & \V[\ltextordmasculine]\textordmasculine \\
\V\textbackslash       & \V\textparagraph       \\
\V\textbar             & \V\textperiodcentered  \\
\V\textbraceleft       & \V\textquestiondown    \\
\V\textbraceright      & \V\textquotedblleft    \\
\V\textbullet          & \V\textquotedblright   \\
\V[\ltextcopyright]\textcopyright
                       & \V\textquoteleft       \\
\V\textdagger          & \V\textquoteright      \\
\V\textdaggerdbl       & \V[\ltextregistered]\textregistered      \\
\V\textdollar          & \V\textsection         \\
\V\textellipsis        & \V\textsterling        \\
\V\textemdash          & \V[\ltexttrademark]\texttrademark        \\
\V\textendash          & \V\textunderscore      \\
\V\textexclamdown      & \V\textvisiblespace    \\
\V\textgreater         \\
\end{tabular}

\bigskip
\twosymbolmessage
\end{symtable}


\begin{symtable}{Non-ASCII Letters (Excluding Accented Letters)}
\index{letters>non-ASCII}\index{ASCII}
\label{non-ascii}
\begin{tabular}{*4{ll@{\qquad}}ll}
\K\aa      & \Ks\DH     & \K\L       & \K\o       & \K\ss      \\
\K\AA      & \Ks\dh     & \K\l       & \K\O       & \K\SS      \\
\K\AE      & \Ks\DJ     & \Ks\NG     & \K\OE      & \Ks\TH     \\
\K\ae      & \Ks\dj     & \Ks\ng     & \K\oe      & \Ks\th     \\
\end{tabular}

\bigskip
\begin{tablenote}[*]
  Not available in the OT1 font encoding.  Use the \pkgname{fontenc}
  package to select an alternate font encoding, such as T1.
\end{tablenote}
\end{symtable}


\begin{symtable}[FC]{Letters Used to Typeset African Languages}
\index{alphabets>African}
\begin{tabular}{*6{ll@{\qquad}}ll}
\Jf\B{D} & \Jf\m{c} & \Jf\m{f} & \Jf\m{k} & \Jf\M{t}     & \Jf\m{Z} \\
\Jf\B{d} & \Jf\m{D} & \Jf\m{F} & \Jf\m{N} & \Jf\M{T}     & \Jf\T{E} \\
\Jf\B{H} & \Jf\M{d} & \Jf\m{G} & \Jf\m{n} & \Jf\m{t}     & \Jf\T{e} \\
\Jf\B{h} & \Jf\M{D} & \Jf\m{g} & \Jf\m{o} & \Jf\m{T}     & \Jf\T{O} \\
\Jf\B{t} & \Jf\m{d} & \Jf\m{I} & \Jf\m{O} & \Jf\m{u}$^*$ & \Jf\T{o} \\
\Jf\B{T} & \Jf\m{E} & \Jf\m{i} & \Jf\m{P} & \Jf\m{U}$^*$ \\
\Jf\m{b} & \Jf\m{e} & \Jf\m{J} & \Jf\m{p} & \Jf\m{Y}     \\
\Jf\m{B} & \Jf\M{E} & \Jf\m{j} & \Jf\m{s} & \Jf\m{y}     \\
\Jf\m{C} & \Jf\M{e} & \Jf\m{K} & \Jf\m{S} & \Jf\m{z}     \\
\end{tabular}

\bigskip
\begin{tablenote}
  These characters all need the T4 font encoding, which is provided
  by the \FC\ package.
\end{tablenote}

\bigskip
\begin{tablenote}[*]
  \verb|\m{v}| and \verb|\m{V}| are synonyms for \verb|\m{u}| and
  \verb|\m{U}|.
\end{tablenote}
\end{symtable}


\begin{symtable}{Punctuation Marks Not Found in OT1}
\index{punctuation}
\label{punc-no-OT1}
\begin{tabular}{*8l}
\Kt\guillemotleft  & \Kt\guilsinglleft & \Kt\quotedblbase & \Kt\textquotedbl \\
\Kt\guillemotright & \Kt\guilsinglright & \Kt\quotesinglbase \\
\end{tabular}

\bigskip
\begin{tablenote}
  To get these symbols, use the \pkgname{fontenc} package to select an
  alternate font encoding, such as~T1.
\end{tablenote}
\end{symtable}


\begin{symtable}[PI]{\PI\ Decorative Punctuation Marks}
\index{punctuation}
\label{pi-punctuation}
\begin{tabular}{*5{ll}}
\Tp{123} & \Tp{125} & \Tp{161} & \Tp{163} \\
\Tp{124} & \Tp{126} & \Tp{162} \\
\end{tabular}
\end{symtable}


\begin{symtable}{Text-Mode Accents}
\index{accents}
\label{text-accents}
\ifFC
  \begin{tabular}{*3{ll@{\hspace*{3em}}}ll}
  \Q\" & \Q\`          & \Q\H          & \Q\u \\
  \Q\' & \Q\b          & \Qt\k$^\dag$  & \Q\v \\
  \Q\. & \Q\c          & \Q\r          & \Q\~ \\
  \Q\= & \Q\d          & \Q\t          \\
  \Q\^ & \Qf\G$^\ddag$ & \Qf\U$^\ddag$ \\
  \end{tabular}
\else
  \begin{tabular}{*3{ll@{\hspace*{3em}}}ll}
  \Q\"    & \Q\^    & \Q\d         & \Q\t    \\
  \Q\'    & \Q\`    & \Q\H         & \Q\u    \\
  \Q\.    & \Q\b    & \Qt\k$^\dag$ & \Q\v    \\
  \Q\=    & \Q\c    & \Q\r         & \Q\~    \\
  \end{tabular}
\fi    % FC test
\par\medskip
\begin{tabular}{ll@{\hspace*{3em}}ll}
\Q\newtie$^*$ & \Q\textcircled
\end{tabular}

\bigskip
\begin{tablenote}[*]
  Requires the \TC\ package.
\end{tablenote}

\medskip
\begin{tablenote}[\dag]
  Not available in the OT1 font encoding.  Use the \pkgname{fontenc}
  package to select an alternate font encoding, such as T1.
\end{tablenote}

\ifFC
\medskip
\begin{tablenote}[\ddag]
  Requires the T4 font encoding, provided by the \FC\ package.
\end{tablenote}
\fi

\bigskip
\begin{tablenote}
  Also note the existence of \cmd{\i} and \cmd{\j}, which produce
  dotless versions of ``i'' and ``j'' (viz., ``\i'' and ``\j'').  These
  are useful when the accent is supposed to replace the dot.  For
  example, ``\verb|na\"{\i}ve|'' produces a correct ``na\"{\i}ve'',
  while ``\verb|na\"{i}ve|'' would yield the rather odd-looking
  ``na\"{i}ve''.  (``\verb|na\"{i}ve|'' \emph{does} work in encodings
  other than OT1, however.)
\end{tablenote}
\end{symtable}


\begin{longsymtable}[TIPA]{\TIPA\ Text-Mode Accents}
\ltindex{accents}
\label{tipa-accents}
\renewcommand{\arraystretch}{1.25}  % Keep high and low accents from touching.
\begin{longtable}{ll}
\multicolumn{2}{l}{\small\textit{(continued from previous page)}} \\[3ex]
\endhead
\endfirsthead
\\[3ex]
\multicolumn{2}{r}{\small\textit{(continued on next page)}}
\endfoot
\endlastfoot
\Q\textacutemacron      \\
\Q\textacutewedge       \\
\Q\textadvancing        \\
\Q\textbottomtiebar     \\
\Q\textbrevemacron      \\
\Q\textcircumacute      \\
\Q\textcircumdot        \\
\Q\textdotacute         \\
\Q\textdotbreve         \\
\Q\textdoublegrave      \\
\Q\textdoublevbaraccent \\
\Q\textgravecircum      \\
\Q\textgravedot         \\
\Q\textgravemid         \\
\Q\textinvsubbridge     \\
\Q\textlowering         \\
\Q\textmidacute         \\
\Q\textovercross        \\
\Q\textoverw            \\
\Q\textpolhook          \\
\Q\textraising          \\
\Q\textretracting       \\
\Q\textringmacron       \\
\Q\textroundcap         \\
\Q\textseagull          \\
\Q\textsubarch          \\
\Q\textsubbar           \\
\Q\textsubbridge        \\
\Q\textsubdot           \\
\Q\textsublhalfring     \\
\Q\textsubplus          \\
\Q\textsubrhalfring     \\
\Q\textsubring          \\
\Q\textsubsquare        \\
\Q\textsubtilde         \\
\Q\textsubumlaut        \\
\Q\textsubw             \\
\Q\textsubwedge         \\
\Q\textsuperimposetilde \\
\Q\textsyllabic         \\
\Q\texttildedot         \\
\Q\texttoptiebar        \\
\Q\textvbaraccent
\end{longtable}

\begin{tablenote}
  \TIPA\ defines shortcut sequences for many of the above.  See the
  \TIPA\ documentation for more information.
\end{tablenote}
\end{longsymtable}


\begin{symtable}[WIPA]{\WIPA\ Text-Mode Accents}
\index{accents}
\label{wipa-accents}
\renewcommand{\arraystretch}{1.25}  % Keep high and low accents from touching.
\begin{tabular}{ll}
\Q\dental \\
\Q\underarch
\end{tabular}
\end{symtable}


\begin{symtable}[WIPA]{\WIPA\ Diacritics}
\index{accents}
\label{wipa-diacritics}
\renewcommand{\arraystretch}{1.25}  % Keep high and low accents from touching.
\begin{tabular}{*5{ll}}
\K\ain        & \K\leftp    & \K\overring   & \K\stress     & \K\underwedge \\
\K\corner     & \K\leftt    & \K\polishhook & \K\syllabic   & \K\upp        \\
\K\downp      & \K\length   & \K\rightp     & \K\underdots  & \K\upt        \\
\K\downt      & \K\midtilde & \K\rightt     & \K\underring  \\
\K\halflength & \K\open     & \K\secstress  & \K\undertilde \\
\end{tabular}

\bigskip
\begin{tablenote}
  The \WIPA\ package defines all of the above as ordinary characters,
  not as accents.  However, it does provide \cmd{\diatop} and
  \cmd{\diaunder} commands, which are used to compose diacritics with
  other characters.  For example, \verb+\diatop[\overring|a]+ produces
  ``\diatop[\overring|a]'', and \verb+\diaunder[\underdots|a]+
  produces ``\diaunder[\underdots|a]''.  See the \WIPA\ documentation
  for more information.
\end{tablenote}
\end{symtable}


\begin{symtable}{\TC\ Diacritics}
\index{accents}
\label{tc-accent-chars}
\begin{tabular}{*3{ll}}
\K\textacutedbl      & \K\textasciicaron    & \K\textasciimacron \\
\K\textasciiacute    & \K\textasciidieresis & \K\textgravedbl    \\
\K\textasciibreve    & \K\textasciigrave    & \K\texttildelow    \\
\end{tabular}

\bigskip

\begin{tablenote}
  The \TC\ package defines all of the above as ordinary characters,
  not as accents.
\end{tablenote}
\end{symtable}


\begin{symtable}{\TC\ Currency Symbols}
\idxboth{currency}{symbols}
\label{tc-currency}
\begin{tabular}{*4{ll}}
\K\textbaht          & \K\textdollar         & \K\textguarani  & \K\textwon \\
\K\textcent          & \K\textdollaroldstyle & \K\textlira     & \K\textyen \\
\K\textcentoldstyle  & \K\textdong           & \K\textnaira    \\
\K\textcolonmonetary & \K\texteuro           & \K\textpeso     \\
\K\textcurrency      & \K\textflorin         & \K\textsterling \\
\end{tabular}
\end{symtable}


\begin{symtable}[MARV]{\MARV\ Currency Symbols}
\idxboth{currency}{symbols}\index{euro signs}
\label{marv-currency}
\begin{tabular}{*4{ll}ll}
\K\Denarius   & \K\EUR    & \K\EURdig   & \K\EURtm      & \K\Pfund      \\
\K\Ecommerce  & \K\EURcr  & \K\EURhv    & \K\EyesDollar & \K\Shilling   \\
\end{tabular}

\bigskip
\begin{tablenote}
  Note that:

  \begin{itemize}
    \item \texttt{\string\Deleatur}\indexcommand{\Deleatur} is another
          macro name for \texttt{\string\Denarius}.
    \item The different euro signs are meant to be compatible with
          different fonts---Courier\index{Courier} (\texttt{\string\EURcr}),
          Helvetica\index{Helvetica} (\texttt{\string\EURhv}),
          Times\index{Times} (\texttt{\string\EURtm}), and the \MARV\ digits
          listed in Table~\ref{marv-math} (\texttt{\string\EURdig}).
  \end{itemize}
\end{tablenote}
\end{symtable}


\begin{symtable}{\TC\ Legal Symbols}
\label{tc-legal}
\begin{tabular}{*2{lll@{\qquad}}lll}
\V\textcircledP & \V[\ltextcopyright]\textcopyright   & \V\textservicemark \\
\V\textcopyleft & \V[\ltextregistered]\textregistered & \V[\ltexttrademark]\texttrademark \\
\end{tabular}

\bigskip
\twosymbolmessage
\end{symtable}


\begin{symtable}{\TC\ Old-Style Numerals}
\idxboth{old-style}{digits}
\label{old-style-nums}
\begin{tabular}{*3{ll}}
\K\textzerooldstyle  & \K\textfouroldstyle  & \K\texteightoldstyle \\
\K\textoneoldstyle   & \K\textfiveoldstyle  & \K\textnineoldstyle  \\
\K\texttwooldstyle   & \K\textsixoldstyle   \\
\K\textthreeoldstyle & \K\textsevenoldstyle \\
\end{tabular}

\bigskip
\begin{tablenote}
  Rather than use the bulky \cmd{\textoneoldstyle},
  \cmd{\texttwooldstyle}, etc.\ commands shown above, consider using
  \cmd{\oldstylenums}\verb|{|$\ldots$\verb|}| to typeset an old-style
  number.
\end{tablenote}
\end{symtable}


\begin{symtable}[WASY]{\WASY\ Phonetic Symbols}
\idxboth{phonetic}{symbols}
\idxboth{linguistic}{symbols}
\index{alphabets>phonetic}
\label{wasy-phonetics}
\begin{tabular}{*8l}
\K\DH             & \K\dh             & \K\openo          \\
\K\Thorn          & \K\inve           & \K\thorn          \\
\end{tabular}
\end{symtable}


\begin{longsymtable}[TIPA]{\TIPA\ Phonetic Symbols}
\ltidxboth{phonetic}{symbols}
\ltidxboth{linguistic}{symbols}
\ltindex{alphabets>phonetic}
\label{tipa-phonetic}
\begin{longtable}{*3{ll}}
\multicolumn{6}{l}{\small\textit{(continued from previous page)}} \\[3ex]
\endhead
\endfirsthead
\\[3ex]
\multicolumn{6}{r}{\small\textit{(continued on next page)}}
\endfoot
\endlastfoot
\K\textbabygamma       & \K\textglotstop        & \K\textrtaill          \\
\K\textbarb            & \K\texthalflength      & \K\textrtailn          \\
\K\textbarc            & \K\texthardsign        & \K\textrtailr          \\
\K\textbard            & \K\texthooktop         & \K\textrtails          \\
\K\textbardotlessj     & \K\texthtb             & \K\textrtailt          \\
\K\textbarg            & \K\texthtbardotlessj   & \K\textrtailz          \\
\K\textbarglotstop     & \K\texthtc             & \K\textrthook          \\
\K\textbari            & \K\texthtd             & \K\textsca             \\
\K\textbarl            & \K\texthtg             & \K\textscb             \\
\K\textbaro            & \K\texthth             & \K\textsce             \\
\K\textbarrevglotstop  & \K\texththeng          & \K\textscg             \\
\K\textbaru            & \K\texthtk             & \K\textsch             \\
\K\textbeltl           & \K\texthtp             & \K\textschwa           \\
\K\textbeta            & \K\texthtq             & \K\textsci             \\
\K\textbullseye        & \K\texthtscg           & \K\textscj             \\
\K\textceltpal         & \K\texthtt             & \K\textscl             \\
\K\textchi             & \K\texthvlig           & \K\textscn             \\
\K\textcloseepsilon    & \K\textinvglotstop     & \K\textscoelig         \\
\K\textcloseomega      & \K\textinvscr          & \K\textscomega         \\
\K\textcloserevepsilon & \K\textiota            & \K\textscq             \\
\K\textcommatailz      & \K\textlambda          & \K\textscr             \\
\K\textcorner          & \K\textlengthmark      & \K\textscripta         \\
\K\textcrb             & \K\textlhookt          & \K\textscriptv         \\
\K\textcrd             & \K\textlhti            & \K\textscu             \\
\K\textcrg             & \K\textlhtlongi        & \K\textscy             \\
\K\textcrh             & \K\textlonglegr        & \K\textsecstress       \\
\K\textcrinvglotstop   & \K\textlptr            & \K\textsoftsign        \\
\K\textcrlambda        & \K\textltailm          & \K\textstretchc        \\
\K\textcrtwo           & \K\textltailn          & \K\texttctclig         \\
\K\textctc             & \K\textltilde          & \K\textteshlig         \\
\K\textctd             & \K\textlyoghlig        & \K\texttheta           \\
\K\textctdctzlig       & \K\textnrleg           & \K\textthorn           \\
\K\textctesh           & \K\textObardotlessj    & \K\texttslig           \\
\K\textctj             & \K\textOlyoghlig       & \K\textturna           \\
\K\textctn             & \K\textomega           & \K\textturncelig       \\
\K\textctt             & \K\textopencorner      & \K\textturnh           \\
\K\textcttctclig       & \K\textopeno           & \K\textturnk           \\
\K\textctyogh          & \K\textpalhook         & \K\textturnlonglegr    \\
\K\textctz             & \K\textphi             & \K\textturnm           \\
\K\textdctzlig         & \K\textpipe            & \K\textturnmrleg       \\
\K\textdoublebaresh    & \K\textprimstress      & \K\textturnr           \\
\K\textdoublebarpipe   & \K\textraiseglotstop   & \K\textturnrrtail      \\
\K\textdoublebarslash  & \K\textraisevibyi      & \K\textturnscripta     \\
\K\textdoublepipe      & \K\textramshorns       & \K\textturnt           \\
\K\textdoublevertline  & \K\textrectangle       & \K\textturnv           \\
\K\textdownstep        & \K\textrevapostrophe   & \K\textturnw           \\
\K\textdyoghlig        & \K\textreve            & \K\textturny           \\
\K\textdzlig           & \K\textrevepsilon      & \K\textupsilon         \\
\K\textepsilon         & \K\textrevglotstop     & \K\textupstep          \\
\K\textesh             & \K\textrevyogh         & \K\textvertline        \\
\K\textfishhookr       & \K\textrhookrevepsilon & \K\textvibyi           \\
\K\textg               & \K\textrhookschwa      & \K\textvibyy           \\
\K\textgamma           & \K\textrhoticity       & \K\textwynn            \\
\K\textglobfall        & \K\textrptr            & \K\textyogh            \\
\K\textglobrise        & \K\textrtaild          \\
\end{longtable}

\begin{tablenote}
  \TIPA\ defines shortcut characters for many of the above.  It also
  defines a command \cmd{\tone} for denoting tone letters (pitches).
  See the \TIPA\ documentation for more information.
\end{tablenote}
\end{longsymtable}


\begin{longsymtable}[WIPA]{\WIPA\ Phonetic Symbols}
\ltidxboth{phonetic}{symbols}
\ltidxboth{linguistic}{symbols}
\ltindex{alphabets>phonetic}
\label{wipa-phonetic}
\begin{longtable}{*4{ll}}
\multicolumn{8}{l}{\small\textit{(continued from previous page)}} \\[3ex]
\endhead
\endfirsthead
\\[3ex]
\multicolumn{8}{r}{\small\textit{(continued on next page)}}
\endfoot
\endlastfoot
\K\babygamma        & \K\eng            & \K\labdentalnas     & \K\schwa            \\
\K\barb             & \K\er             & \K\latfric          & \K\sci              \\
\K\bard             & \K\esh            & \K\legm             & \K\scn              \\
\K\bari             & \K[\WSUeth]\eth   & \K\legr             & \K\scr              \\
\K\barl             & \K\flapr          & \K\lz               & \K\scripta          \\
\K[\WSUbaro]\baro   & \K\glotstop       & \K\nialpha          & \K\scriptg          \\
\K\barp             & \K\hookb          & \K\nibeta           & \K\scriptv          \\
\K\barsci           & \K\hookd          & \K\nichi            & \K\scu              \\
\K\barscu           & \K\hookg          & \K\niepsilon        & \K\scy              \\
\K\baru             & \K\hookh          & \K\nigamma          & \K\slashb           \\
\K\clickb           & \K\hookheng       & \K\niiota           & \K\slashc           \\
\K\clickc           & \K\hookrevepsilon & \K\nilambda         & \K\slashd           \\
\K\clickt           & \K\hv             & \K\niomega          & \K\slashu           \\
\K\closedniomega    & \K\inva           & \K\niphi            & \K\taild            \\
\K\closedrevepsilon & \K\invf           & \K\nisigma          & \K\tailinvr         \\
\K\crossb           & \K\invglotstop    & \K\nitheta          & \K\taill            \\
\K\crossd           & \K\invh           & \K\niupsilon        & \K\tailn            \\
\K\crossh           & \K\invlegr        & \K\nj               & \K\tailr            \\
\K\crossnilambda    & \K\invm           & \K\oo               & \K\tails            \\
\K\curlyc           & \K\invr           & \K[\WSUopeno]\openo & \K\tailt            \\
\K\curlyesh         & \K\invscr         & \K\reve             & \K\tailz            \\
\K\curlyyogh        & \K\invscripta     & \K\reveject         & \K\tesh             \\
\K\curlyz           & \K\invv           & \K\revepsilon       & \K[\WSUthorn]\thorn \\
\K\dlbari           & \K\invw           & \K\revglotstop      & \K\tildel           \\
\K\dz               & \K\invy           & \K\scd              & \K\yogh             \\
\K\ejective         & \K\ipagamma       & \K\scg              \\
\end{longtable}
\end{longsymtable}


\begin{symtable}{Miscellaneous \TC\ Symbols}
\index{musical notes}
\label{tc-misc}
\begin{tabular}{lll@{\qquad}lll}
\V\textasteriskcentered     & \V\textopenbullet           \\
\V\textbardbl               & \V[\ltextordfeminine]\textordfeminine \\
\V\textbigcircle            & \V[\ltextordmasculine]\textordmasculine \\
\V\textblank                & \V\textparagraph            \\
\V\textbrokenbar            & \V\textperiodcentered       \\
\V\textbullet               & \V\textpertenthousand       \\
\V\textdagger               & \V\textperthousand          \\
\V\textdaggerdbl            & \V\textpilcrow              \\
\V\textdblhyphen            & \V\textquotesingle          \\
\V\textdblhyphenchar        & \V\textquotestraightbase    \\
\V\textdiscount             & \V\textquotestraightdblbase \\
\V\textestimated            & \V\textrecipe               \\
\V\textinterrobang          & \V\textreferencemark        \\
\V\textinterrobangdown      & \V\textsection              \\
\V\textmusicalnote          & \V\textthreequartersemdash  \\
\V\textnumero               & \V\texttwelveudash          \\
\end{tabular}

\bigskip
\twosymbolmessage
\end{symtable}


\section{Mathematical symbols}
\label{math-symbols}
\idxboth{mathematical}{symbols}

Most, but not all, of the symbols in this section are math-mode only.
That is, they yield a ``\texttt{Missing~\$ inserted}''\index{Missing \$
inserted=\texttt{Missing~\$ inserted}} error message if not used within
\verb|$|$\ldots$\verb|$|, \verb|\[|$\ldots$\verb|\]|, or another
math-mode environment.  Operators marked as ``variable-sized'' are
taller in displayed formulas, shorter in in-text formulas, and possibly
shorter still when used in various levels of superscripts or subscripts.

\ifcomplete
Alphanumeric symbols (e.g., ``$\!\mathscr{L}\,$'' and ``$\varmathbb{Z}$'')
are usually produced using one of the math alphabets in
Table~\ref{alphabets} rather than with an explicit symbol command.  Look
there first if you need a symbol for a transform, number set, or some
other alphanumeric.
\fi

The various text-mode symbols defined by the \TC\ package are made
available in math mode through the \pkgname{mathcomp} package.


\bigskip

\begin{symtable}{Binary Operators}
\idxboth{binary}{operators}
\label{bin}
\begin{tabular}{*4{ll}}
\X\amalg           & \X\cup          & \X\oplus    & \X\times           \\
\X\ast             & \X\dagger       & \X\oslash   & \X\triangleleft    \\
\X\bigcirc         & \X\ddagger      & \X\otimes   & \X\triangleright   \\
\X\bigtriangledown & \X\diamond      & \X\pm       & \X\unlhd$^*$       \\
\X\bigtriangleup   & \X\div          & \X\rhd$^*$  & \X\unrhd$^*$       \\
\X\bullet          & \X\lhd$^*$      & \X\setminus & \X\uplus           \\
\X\cap             & \X\mp           & \X\sqcap    & \X\vee             \\
\X\cdot            & \X\odot         & \X\sqcup    & \X\wedge           \\
\X\circ            & \X\ominus       & \X\star     & \X\wr              \\
\end{tabular}

\bigskip
\notpredefinedmessage
\end{symtable}


\begin{symtable}{Variable-sized Math Operators}
\idxboth{variable-sized}{symbols}
\index{integrals}
\label{op}
\renewcommand{\arraystretch}{1.75}  % Keep tall symbols from touching.
\begin{tabular}{*3{l@{$\:$}ll@{\qquad}}l@{$\:$}ll}
\R\bigcap    & \R\bigotimes & \R\bigwedge  & \R\prod      \\
\R\bigcup    & \R\bigsqcup  & \R\coprod    & \R\sum       \\
\R\bigodot   & \R\biguplus  & \R\int       \\
\R\bigoplus  & \R\bigvee    & \R\oint      \\
\end{tabular}
\end{symtable}


\begin{symtable}[AMS]{\AMS\ Binary Operators}
\idxboth{binary}{operators}
\label{ams-bin}
\begin{tabular}{*3{ll}}
\X\barwedge        & \X\circledcirc     & \X\intercal        \\
\X\boxdot          & \X\circleddash     & \X\leftthreetimes  \\
\X\boxminus        & \X\Cup             & \X\ltimes          \\
\X\boxplus         & \X\curlyvee        & \X\rightthreetimes \\
\X\boxtimes        & \X\curlywedge      & \X\rtimes          \\
\X\Cap             & \X\divideontimes   & \X\smallsetminus   \\
\X\centerdot       & \X\dotplus         & \X\veebar          \\
\X\circledast      & \X\doublebarwedge  \\
\end{tabular}
\end{symtable}


\begin{symtable}[ST]{\ST\ Binary Operators}
\idxboth{binary}{operators}
\label{st-bin}
\begin{tabular}{*3{ll}}
\X\baro                & \X\interleave          & \X\varoast             \\
\X\bbslash             & \X\leftslice           & \X\varobar             \\
\X\binampersand        & \X\merge               & \X\varobslash          \\
\X\bindnasrepma        & \X\minuso              & \X\varocircle          \\
\X\boxast              & \X\moo                 & \X\varodot             \\
\X\boxbar              & \X\nplus               & \X\varogreaterthan     \\
\X\boxbox              & \X\obar                & \X\varolessthan        \\
\X\boxbslash           & \X\oblong              & \X\varominus           \\
\X\boxcircle           & \X\obslash             & \X\varoplus            \\
\X\boxdot              & \X\ogreaterthan        & \X\varoslash           \\
\X\boxempty            & \X\olessthan           & \X\varotimes           \\
\X\boxslash            & \X\ovee                & \X\varovee             \\
\X\curlyveedownarrow   & \X\owedge              & \X\varowedge           \\
\X\curlyveeuparrow     & \X\rightslice          & \X\vartimes            \\
\X\curlywedgedownarrow & \X\sslash              & \X\Ydown               \\
\X\curlywedgeuparrow   & \X\talloblong          & \X\Yleft               \\
\X\fatbslash           & \X\varbigcirc          & \X\Yright              \\
\X\fatsemi             & \X\varcurlyvee         & \X\Yup                 \\
\X\fatslash            & \X\varcurlywedge       \\
\end{tabular}
\end{symtable}


\begin{symtable}[ST]{Variable-sized \ST\ Math Operators}
\idxboth{variable-sized}{symbols}
\label{st-large}
\renewcommand{\arraystretch}{1.75}  % Keep tall symbols from touching.
\begin{tabular}{*2{l@{$\:$}ll@{\qquad}}l@{$\:$}ll}
\R\bigbox        & \R\biginterleave & \R\bigsqcap        \\
\R\bigcurlyvee   & \R\bignplus      & \R[\STbigtriangledown]\bigtriangledown \\
\R\bigcurlywedge & \R\bigparallel   & \R[\STbigtriangleup]\bigtriangleup   \\
\end{tabular}
\end{symtable}


\begin{symtable}[WASY]{Variable-sized \WASY\ Math Operators}
\idxboth{variable-sized}{symbols}
\index{integrals}
\label{wasy-large}
\renewcommand{\arraystretch}{1.75}  % Keep tall symbols from touching.
\begin{tabular}{*2{l@{$\:$}ll@{\qquad}}l@{$\:$}ll}
\R\iiint   & \R\oiint   & \R\varoint \\
\R\iint    & \R\varint  \\
\end{tabular}
\end{symtable}


\begin{symtable}[TX]{\TXPX\ Binary Operators}
\idxboth{binary}{operators}
\label{txpx-bin}
\begin{tabular}{*3{ll}}
\X\circledbar    & \X\circledwedge  & \X\medcirc       \\
\X\circledbslash & \X\invamp        & \X\sqcapplus     \\
\X\circledvee    & \X\medbullet     & \X\sqcupplus     \\
\end{tabular}
\end{symtable}


\begin{symtable}[TX]{Variable-sized \TXPX\ Math Operators}
\idxboth{variable-sized}{symbols}
\index{integrals}
\label{txpx-large}
\renewcommand{\arraystretch}{1.85}  % Keep tall symbols from touching.
\ifAMS\def\alsoAMS{$^*$}\else\def\alsoAMS{}\fi
\ifWASY\def\alsoWASY{$^\dag$}\else\def\alsoWASY{}\fi
\def\alsoboth{}\ifAMS\ifWASY\def\alsoboth{$^{*,\dag}$}\fi\fi

\begin{tabular}{l@{$\:$}ll@{\qquad}l@{$\:$}ll}
\R\bigsqcapplus          & \R\ointclockwise         \\
\R\bigsqcupplus          & \R\ointctrclockwise      \\
\R\fint                  & \R\sqiiint               \\
\R\idotsint\alsoAMS      & \R\sqiint                \\
\R\iiiint\alsoAMS        & \R\sqint                 \\
\R\iiint\alsoboth        & \R\varoiiintclockwise    \\
\R\iint\alsoboth         & \R\varoiiintctrclockwise \\
\R\oiiintclockwise       & \R\varoiintclockwise     \\
\R\oiiintctrclockwise    & \R\varoiintctrclockwise  \\
\R\oiiint                & \R\varointclockwise      \\
\R\oiintclockwise        & \R\varointctrclockwise   \\
\R\oiintctrclockwise     & \R\varprod               \\
\R\oiint\alsoWASY        \\
\end{tabular}

\ifAMS
  \bigskip
  \begin{tablenote}[*]
    Also defined by \pkgname{amsmath}.
  \end{tablenote}
\fi

\ifWASY
  \medskip
  \begin{tablenote}[\dag]
    Also defined by \WASY.
  \end{tablenote}
\fi
\end{symtable}


\begin{symtable}{Relation Symbols}
\idxboth{relational}{symbols}
\label{rel}
\begin{tabular}{*4{ll}}
\X\approx       & \X\in           & \X\prec         & \X\subset       \\
\X\asymp        & \X\Join$^*$     & \X\preceq       & \X\subseteq     \\
\X\bowtie       & \X\leq          & \X\propto       & \X\succ         \\
\X\cong         & \X\ll           & \X\sim          & \X\succeq       \\
\X\dashv        & \X\mid          & \X\simeq        & \X\supset       \\
\X\doteq        & \X\models       & \X\smile        & \X\supseteq     \\
\X\equiv        & \X\neq          & \X\sqsubset$^*$ & \X\vdash        \\
\X\frown        & \X\ni           & \X\sqsubseteq   \\
\X\geq          & \X\parallel     & \X\sqsupset$^*$ \\
\X\gg           & \X\perp         & \X\sqsupseteq   \\
\end{tabular}

\bigskip
\notpredefinedmessage
\end{symtable}


\begin{symtable}[AMS]{\AMS\ Binary Relations}
\index{binary relations}
\index{relational symbols>binary}
\label{ams-rel}
\begin{tabular}{*3{ll}}
\X\approxeq           & \X\gtrdot             & \X\smallsmile         \\
\X\backepsilon        & \X\gtreqless          & \X\sqsubset           \\
\X\backsim            & \X\gtreqqless         & \X\sqsupset           \\
\X\backsimeq          & \X\gtrless            & \X\Subset             \\
\X\because            & \X\gtrsim             & \X\subseteqq          \\
\X\between            & \X\leqq               & \X\succapprox         \\
\X\blacktriangleleft  & \X\leqslant           & \X\succcurlyeq        \\
\X\blacktriangleright & \X\lessapprox         & \X\succsim            \\
\X\Bumpeq             & \X\lessdot            & \X\Supset             \\
\X\bumpeq             & \X\lesseqgtr          & \X\supseteqq          \\
\X\circeq             & \X\lesseqqgtr         & \X\therefore          \\
\X\curlyeqprec        & \X\lessgtr            & \X\thickapprox        \\
\X\curlyeqsucc        & \X\lesssim            & \X\thicksim           \\
\X\doteqdot           & \X\lll                & \X\trianglelefteq     \\
\X\eqcirc             & \X\pitchfork          & \X\triangleq          \\
\X\eqslantgtr         & \X\precapprox         & \X\trianglerighteq    \\
\X\eqslantless        & \X\preccurlyeq        & \X\varpropto          \\
\X\fallingdotseq      & \X\precsim            & \X\vartriangleleft    \\
\X\geqq               & \X\risingdotseq       & \X\vartriangleright   \\
\X\geqslant           & \X\shortmid           & \X\Vdash              \\
\X\ggg                & \X\shortparallel      & \X\vDash              \\
\X\gtrapprox          & \X\smallfrown         & \X\Vvdash             \\
\end{tabular}
\end{symtable}

\begin{symtable}[AMS]{\AMS\ Negated Binary Relations}
\index{binary relations>negated}
\index{relational symbols>negated binary}
\label{ams-nrel}
\begin{tabular}{*3{ll}}
\X\gnapprox         & \X\nleqslant        & \X\ntrianglerighteq \\
\X\gneq             & \X\nless            & \X\nvdash           \\
\X\gneqq            & \X\nmid             & \X\nvDash           \\
\X\gnsim            & \X\nparallel        & \X\nVDash           \\
\X\gvertneqq        & \X\nprec            & \X\precnapprox      \\
\X\lnapprox         & \X\npreceq          & \X\precnsim         \\
\X\lneq             & \X\nshortmid        & \X\subsetneq        \\
\X\lneqq            & \X\nshortparallel   & \X\subsetneqq       \\
\X\lnsim            & \X\nsim             & \X\succnapprox      \\
\X\lvertneqq        & \X\nsubseteq        & \X\succnsim         \\
\X\ncong            & \X\nsucc            & \X\supsetneq        \\
\X\ngeq             & \X\nsucceq          & \X\supsetneqq       \\
\X\ngeqq            & \X\nsupseteq        & \X\varsubsetneq     \\
\X\ngeqslant        & \X\nsupseteqq       & \X\varsubsetneqq    \\
\X\ngtr             & \X\ntriangleleft    & \X\varsupsetneq     \\
\X\nleq             & \X\ntrianglelefteq  & \X\varsupsetneqq    \\
\X\nleqq            & \X\ntriangleright   \\
\end{tabular}
\end{symtable}


\begin{symtable}[ST]{\ST\ Binary Relations}
\index{binary relations}
\index{relational symbols>binary}
\label{st-rel}
\begin{tabular}{*3{ll}}
\X\inplus               & \X\subsetpluseq         & \X\trianglelefteqslant  \\
\X\niplus               & \X\supsetplus           & \X\trianglerighteqslant \\
\X\subsetplus           & \X\supsetpluseq         \\
\end{tabular}
\end{symtable}


\begin{symtable}[ST]{\ST\ Negated Binary Relations}
\index{binary relations>negated}\index{relational symbols>negated binary}
\label{st-nrel}
\begin{tabular}{*8l}
\X\ntrianglelefteqslant &\X\ntrianglerighteqslant
\end{tabular}
\end{symtable}


\begin{longsymtable}[TX]{\TXPX\ Binary Relations}
\ltindex{binary relations}
\ltindex{relational symbols>binary}
\label{txpx-rel}
\begin{longtable}{*3{ll}}
\multicolumn{6}{l}{\small\textit{(continued from previous page)}} \\[3ex]
\endhead
\endfirsthead
\\[3ex]
\multicolumn{6}{r}{\small\textit{(continued on next page)}}
\endfoot
\endlastfoot
\X\boxdotLeft           & \X\lrtimes              & \X\npreceqq             \\
\X\boxdotleft           & \X\Mappedfromchar       & \X\nprecsim             \\
\X\boxdotright          & \X\mappedfromchar       & \X\nsimeq               \\
\X\boxdotRight          & \X\mmappedfromchar      & \X\nsqsubset            \\
\X\boxleft              & \X\Mmappedfromchar      & \X\nsqsubseteq          \\
\X\boxLeft              & \X\mmapstochar          & \X\nsqsupset            \\
\X\boxRight             & \X\Mmapstochar          & \X\nsqsupseteq          \\
\X\boxright             & \X\multimapboth         & \X\nSubset              \\
\X\circleddotleft       & \X\multimapbothvert     & \X\nsubseteqq           \\
\X\circleddotright      & \X\multimapdot          & \X\nsuccapprox          \\
\X\circledgtr           & \X\multimapdotboth      & \X\nsucccurlyeq         \\
\X\circledless          & \X\multimapdotbothA     & \X\nsucceqq             \\
\X\circleleft           & \X\multimapdotbothAvert & \X\nsuccsim             \\
\X\circleright          & \X\multimapdotbothB     & \X\nSupset              \\
\X\colonapprox          & \X\multimapdotbothBvert & \X\nthickapprox         \\
\X\Colonapprox          & \X\multimapdotbothvert  & \X\ntwoheadleftarrow    \\
\X\coloneq              & \X\multimapdotinv       & \X\ntwoheadrightarrow   \\
\X\Coloneq              & \X\multimapinv          & \X\nvarparallel         \\
\X\coloneqq             & \X\napproxeq            & \X\nvarparallelinv      \\
\X\Coloneqq             & \X\nasymp               & \X\nVdash               \\
\X\colonsim             & \X\nbacksim             & \X\Nwarrow              \\
\X\Colonsim             & \X\nbacksimeq           & \X\openJoin             \\
\X\dashleftrightarrow   & \X\nBumpeq              & \X\opentimes            \\
\X\Diamonddotleft       & \X\nbumpeq              & \X\Perp                 \\
\X\DiamonddotLeft       & \X\Nearrow              & \X\preceqq              \\
\X\Diamonddotright      & \X\nequiv               & \X\precneqq             \\
\X\DiamonddotRight      & \X\ngg                  & \X\rJoin                \\
\X\Diamondleft          & \X\ngtrapprox           & \X\Rrightarrow          \\
\X\DiamondLeft          & \X\ngtrless             & \X\Searrow              \\
\X\Diamondright         & \X\ngtrsim              & \X\strictfi             \\
\X\DiamondRight         & \X\nlessapprox          & \X\strictif             \\
\X\Eqcolon              & \X\nlessgtr             & \X\strictiff            \\
\X\eqcolon              & \X\nlesssim             & \X\succeqq              \\
\X\Eqqcolon             & \X\nll                  & \X\succneqq             \\
\X\eqqcolon             & \X\notin                & \X\Swarrow              \\
\X\eqsim                & \X\notni                & \X\varparallel          \\
\X\leftsquigarrow       & \X\nprecapprox          & \X\varparallelinv       \\
\X\lJoin                & \X\npreccurlyeq         & \X\VvDash               \\
\end{longtable}
\end{longsymtable}


\begin{symtable}{Arrow Symbols}
\index{arrows}
\label{arrow}
\begin{tabular}{*3{ll}}
\X\Downarrow       & \X\longleftarrow           & \X\Rightarrow        \\
\X\downarrow       & \X\Longleftarrow           & \X\rightharpoondown  \\
\X\hookleftarrow   & \X\longleftrightarrow      & \X\rightharpoonup    \\
\X\hookrightarrow  & \X\Longleftrightarrow      & \X\rightleftharpoons \\
\X\leadsto$^*$     & \X\longmapsto              & \X\searrow           \\
\X\Leftarrow       & \X\longrightarrow          & \X\swarrow           \\
\X\leftarrow       & \X\Longrightarrow          & \X\uparrow           \\
\X\leftharpoondown & \X\mapsto                  & \X\Uparrow           \\
\X\leftharpoonup   & \X\nearrow                 & \X\Updownarrow       \\
\X\Leftrightarrow  & \X\nwarrow                 & \X\updownarrow       \\
\X\leftrightarrow  & \X\rightarrow              \\
\end{tabular}

\bigskip
\notpredefinedmessage
\end{symtable}


\begin{symtable}[AMS]{\AMS\ Arrows}
\index{arrows}
\label{ams-arrows}
\begin{tabular}{*3{ll}}
\X\circlearrowleft     & \X\leftleftarrows      & \X\rightleftarrows     \\
\X\circlearrowright    & \X\leftrightarrows     & \X[\AMSrightleftharpoons]\rightleftharpoons \\
\X\curvearrowleft      & \X\leftrightharpoons   & \X\rightrightarrows    \\
\X\curvearrowright     & \X\leftrightsquigarrow & \X\rightsquigarrow     \\
\X\dashleftarrow       & \X\Lleftarrow          & \X\Rsh                 \\
\X\dashrightarrow      & \X\looparrowleft       & \X\twoheadleftarrow    \\
\X\downdownarrows      & \X\looparrowright      & \X\twoheadrightarrow   \\
\X\downharpoonleft     & \X\Lsh                 & \X\upharpoonleft       \\
\X\downharpoonright    & \X\multimap            & \X\upharpoonright      \\
\X\leftarrowtail       & \X\rightarrowtail      & \X\upuparrows          \\
\end{tabular}
\end{symtable}


\begin{symtable}[AMS]{\AMS\ Negated Arrows}
\index{arrows>negated}
\label{ams-narrows}
\begin{tabular}{*3{ll}}
\X\nLeftarrow      & \X\nLeftrightarrow & \X\nRightarrow     \\
\X\nleftarrow      & \X\nleftrightarrow & \X\nrightarrow     \\
\end{tabular}
\end{symtable}


\begin{symtable}[ST]{\ST\ Arrows}
\index{arrows}
\label{st-arrows}
\begin{tabular}{*3{ll}}
\X\leftarrowtriangle      & \X\Mapsfrom           & \X\shortdownarrow  \\
\X\leftrightarroweq       & \X\mapsfrom           & \X\shortleftarrow  \\
\X\leftrightarrowtriangle & \X\Mapsto             & \X\shortrightarrow \\
\X\lightning              & \X\nnearrow           & \X\shortuparrow    \\
\X\Longmapsfrom           & \X\nnwarrow           & \X\ssearrow        \\
\X\longmapsfrom           & \X\rightarrowtriangle & \X\sswarrow        \\
\X\Longmapsto             & \X\rrparenthesis      \\
\end{tabular}
\end{symtable}


\begin{symtable}{Log-like Symbols}
\idxboth{log-like}{symbols}
\label{log}
\begin{tabular}{*8l}
\Z\arccos & \Z\cos  & \Z\csc & \Z\exp & \Z\ker    & \Z\limsup & \Z\min & \Z\sinh \\
\Z\arcsin & \Z\cosh & \Z\deg & \Z\gcd & \Z\lg     & \Z\ln     & \Z\Pr  & \Z\sup  \\
\Z\arctan & \Z\cot  & \Z\det & \Z\hom & \Z\lim    & \Z\log    & \Z\sec & \Z\tan  \\
\Z\arg    & \Z\coth & \Z\dim & \Z\inf & \Z\liminf & \Z\max    & \Z\sin & \Z\tanh
\end{tabular}

\bigskip
\begin{tablenote}
  Calling the above ``symbols'' may be a bit misleading.\footnotemark{}
  Each log-like symbol merely produces the eponymous textual equivalent,
  but with proper surrounding spacing.  See Section~\ref{math-spacing}
  for more information.
\end{tablenote}
\end{symtable}
\footnotetext{Michael\index{Downes, Michael J.} J. Downes prefers the
more general term, ``atomic math objects''.}


\begin{symtable}[AMS]{\AMS\ Log-like Symbols}
\idxboth{log-like}{symbols}
\label{ams-log}
\renewcommand{\arraystretch}{1.5}  % Keep tall symbols from touching.
\begin{tabular}{*2{ll@{\qquad}}ll}
\X\injlim     & \X\varinjlim  & \X\varlimsup  \\
\X\projlim    & \X\varliminf  & \X\varprojlim
\end{tabular}

\bigskip
\begin{tablenote}
  Load the \pkgname{amsmath} package to get these symbols.  See
  Section~\ref{math-spacing} for some additional comments regarding
  log-like symbols.
\end{tablenote}
\end{symtable}


\begin{symtable}{Greek Letters}
\index{Greek}\index{alphabets>Greek}
\label{greek}
\begin{tabular}{*8l}
\X\alpha        &\X\theta       &\X o           &\X\tau         \\
\X\beta         &\X\vartheta    &\X\pi          &\X\upsilon     \\
\X\gamma        &\X\iota        &\X\varpi       &\X\phi         \\
\X\delta        &\X\kappa       &\X\rho         &\X\varphi      \\
\X\epsilon      &\X\lambda      &\X\varrho      &\X\chi         \\
\X\varepsilon   &\X\mu          &\X\sigma       &\X\psi         \\
\X\zeta         &\X\nu          &\X\varsigma    &\X\omega       \\
\X\eta          &\X\xi                                          \\
                                                                \\
\X\Gamma        &\X\Lambda      &\X\Sigma       &\X\Psi         \\
\X\Delta        &\X\Xi          &\X\Upsilon     &\X\Omega       \\
\X\Theta        &\X\Pi          &\X\Phi
\end{tabular}

\bigskip
\begin{tablenote}
  The remaining Greek majuscules\index{majuscules} can be produced with
  ordinary Latin letters.  The symbol ``M'', for instance, is used for
  both an uppercase ``m'' and an uppercase ``$\mu$''.
\end{tablenote}
\end{symtable}


\begin{symtable}[AMS]{\AMS\ Greek Letters}
\index{Greek}\index{alphabets>Greek}
\label{ams-greek}
\begin{tabular}{*4l}
\X\digamma      &\X\varkappa
\end{tabular}
\end{symtable}


\begin{symtable}[TX]{\TXPX\ Upright Greek Letters}
\index{Greek}\index{alphabets>Greek}
\index{upright Greek letters}
\label{txpx-greek}
\begin{tabular}{*4{ll}}
\X\alphaup      & \X\thetaup      & \X\piup         & \X\phiup        \\
\X\betaup       & \X\varthetaup   & \X\varpiup      & \X\varphiup     \\
\X\gammaup      & \X\iotaup       & \X\rhoup        & \X\chiup        \\
\X\deltaup      & \X\kappaup      & \X\varrhoup     & \X\psiup        \\
\X\epsilonup    & \X\lambdaup     & \X\sigmaup      & \X\omegaup      \\
\X\varepsilonup & \X\muup         & \X\varsigmaup   \\
\X\zetaup       & \X\nuup         & \X\tauup        \\
\X\etaup        & \X\xiup         & \X\upsilonup    \\
\end{tabular}
\end{symtable}


\begin{symtable}[TX]{\TXPX\ Variant Latin Letters}
\index{letters>variant Latin}
\label{txpx-variant}
\begin{tabular}{*3{ll@{\qquad}}ll}
\X\varg & \X\varv & \X\varw & \X\vary \\
\end{tabular}

\bigskip
\begin{tablenote}
  Pass the \optname{varg} option to \TXPX\ to replace~$g$, $v$, $w$,
  and~$y$ with~$\varg$, $\varv$, $\varw$, and~$\vary$ in every
  mathematical expression in your document.
\end{tablenote}
\end{symtable}


\begin{symtable}[AMS]{\AMS\ Hebrew Letters}
\index{Hebrew}\index{alphabets>Hebrew}
\label{ams-hebrew}
\begin{tabular}{*6l}
\X\beth &\X\daleth      &\X\gimel
\end{tabular}

\bigskip
\begin{tablenote}
\cmd{\aleph} appears in Table~\vref{ord}.
\end{tablenote}
\end{symtable}


\begin{symtable}{Variable-sized Delimiters}
\index{delimiters>variable-sized}
\label{dels}
\renewcommand{\arraystretch}{1.75}  % Keep tall symbols from touching.
\begin{tabular}{*3{lll@{\qquad}}lll}
\N(             &\N)            &\N\uparrow     &\N\Uparrow     \\
\N{[}           &\N]            &\N\downarrow   &\N\Downarrow   \\
\N\{            &\N\}           &\N\updownarrow &\N\Updownarrow \\
\N\lfloor       &\N\rfloor      &\N\lceil       &\N\rceil       \\
\N\langle       &\N\rangle      &\N/            &\N\backslash   \\
\N|             &\N\|
\end{tabular}

\bigskip
\begin{tablenote}
  When used with \cmd{\left} and \cmd{\right}, these symbols expand
  to the height of the inner math expression.
\end{tablenote}
\end{symtable}


\begin{symtable}{Large, Variable-sized Delimiters}
\index{delimiters>variable-sized}
\label{ldels}
\renewcommand{\arraystretch}{2.5}  % Keep tall symbols from touching.
\begin{tabular}{*3{lll@{\qquad}}lll}
\Y\rmoustache & \Y\lmoustache & \Y\rgroup    & \Y\lgroup \\
\Y\arrowvert  & \Y\Arrowvert  & \Y\bracevert
\end{tabular}

\bigskip
\begin{tablenote}
  These symbols \emph{must} be used with \cmd{\left} and \cmd{\right}.
\end{tablenote}
\end{symtable}


\begin{symtable}[AMS]{\AMS\ Delimiters}
\index{delimiters}
\label{ams-del}
\begin{tabular}{*4{ll}}
\X\ulcorner & \X\urcorner & \X\llcorner & \X\lrcorner
\end{tabular}
\end{symtable}


\begin{symtable}[ST]{\ST\ Delimiters}
\index{delimiters}
\label{st-del}
\begin{tabular}{*4{ll}}
\X\Lbag          &\X\Rbag          &\X\lbag          &\X\rbag    \\
\X\llceil        &\X\rrceil        &\X\llfloor       &\X\rrfloor \\
\X\llparenthesis &\X\rrparenthesis
\end{tabular}
\end{symtable}


\begin{symtable}[ST]{Variable-Sized \ST\ Delimiters}
\index{delimiters>variable-sized}
\label{st-var-del}
\begin{tabular}{lll@{\qquad}lll}
\N\llbracket & \N\rrbracket
\end{tabular}
\end{symtable}


\begin{symtable}{\TC\ Text-Mode Delimiters}
\label{tc-delimiters}
\begin{tabular}{*2{ll}}
\K\textlangle    & \K\textrangle    \\
\K\textlbrackdbl & \K\textrbrackdbl \\
\K\textlquill    & \K\textrquill    \\
\end{tabular}
\end{symtable}


\begin{symtable}{Math-Mode Accents}
\index{accents}
\label{math-accents}
\begin{tabular}{*5{ll}}
\W\acute{a} & \W\breve{a} & \W\ddot{a} & \W\grave{a} & \W\tilde{a} \\
\W\bar{a}   & \W\check{a} & \W\dot{a}  & \W\hat{a}   & \W\vec{a}   \\
\end{tabular}

\bigskip
\begin{tablenote}
  Also note the existence of \cmd{\imath} and \cmd{\jmath}, which
  produce dotless versions of ``\textit{i}'' and ``\textit{j}''.  (See
  Table~\vref{ord}.)  These are useful when the accent is supposed to
  replace the dot.  For example, ``\verb|\hat{\imath}|'' produces a
  correct ``$\,\hat{\imath}\,$'', while ``\verb|\hat{i}|'' would yield
  the rather odd-looking ``\,$\hat{i}\,$''.
\end{tablenote}
\end{symtable}


\begin{symtable}{Some Other Constructions}
\label{other}
\renewcommand{\arraystretch}{1.5}
\begin{tabular}{*4l}
\W\widetilde{abc}       &\W\widehat{abc}                        \\
\W\overleftarrow{abc}   &\W\overrightarrow{abc}                 \\
\W\overline{abc}        &\W\underline{abc}                      \\
\W\overbrace{abc}       &\W\underbrace{abc}                     \\[5pt]
\W\sqrt{abc}            &$\sqrt[n]{abc}$&\verb|\sqrt[n]{abc}|   \\
$f'$&\verb|f'|          &$\frac{abc}{xyz}$&\verb|\frac{abc}{xyz}|
\end{tabular}
\end{symtable}


\begin{symtable}[AMS]{\AMS\ Extensible Arrow Accents}
\index{accents}
\idxboth{extensible}{arrows}
\label{extensible-arrows}
\begin{tabular}{ll@{\qquad}ll@{\qquad}ll}
\W\overleftarrow{a}  & \W\overrightarrow{a}  & \W\overleftrightarrow{a} \\[1ex]
\W\underleftarrow{a} & \W\underrightarrow{a} & \W\underleftrightarrow{a}
\end{tabular}

\bigskip

\begin{tablenote}
  These accents are called ``extensible'' because they stretch to fit
  their argument.  For example, ``\verb|$\underrightarrow{ABCdef}$|''
  produces ``$\underrightarrow{ABCdef}$''.
\end{tablenote}
\end{symtable}


\begin{symtable}{Punctuation Symbols (Math Mode)}
\index{punctuation}
\label{punct}
\begin{tabular}{*{5}{lp{3.2em}}}
\X,     &\X;    &\X\colon$^*$    &\X\ldotp       &\X\cdotp
\end{tabular}

\bigskip
\begin{tablenote}[*]
  While ``\texttt{:}'' is valid in math mode, \cmd{\colon} uses
  different surrounding spacing.  See Section~\ref{math-spacing} and the
  Short Math Guide for \latex~\cite{Downes:smg} for more information on
  math-mode spacing.
\end{tablenote}
\end{symtable}

\begin{symtable}{Miscellaneous \latexE{} Symbols}
\idxboth{miscellaneous}{symbols}
\index{diamonds (suit)}
\index{hearts (suit)}
\index{clubs (suit)}
\index{spades (suit)}
\index{musical notes}
\label{ord}
\begin{tabular}{*4{ll}}
\X\aleph       & \X\ell         & \X\jmath       & \X\spadesuit   \\
\X\angle       & \X\emptyset    & \X\ldots       & \X\surd        \\
\X\backslash   & \X\exists      & \X\mho$^*$     & \X\top         \\
\X\bot         & \X\flat        & \X\nabla       & \X\triangle    \\
\X\Box$^*$     & \X\forall      & \X\natural     & \X\vdots       \\
\X\cdots       & \X\hbar        & \X\neg         & \X\wp          \\
\X\clubsuit    & \X\heartsuit   & \X\partial     \\
\X\ddots       & \X\Im          & \X\prime       \\
\X\Diamond$^*$ & \X\imath       & \X\Re          \\
\X\diamondsuit & \X\infty       & \X\sharp       \\
\end{tabular}

\bigskip
\notpredefinedmessage
\end{symtable}


\begin{symtable}[AMS]{Miscellaneous \AMS\ Symbols}
\idxboth{miscellaneous}{symbols}
\index{stars}
\index{triangles}
\label{ams-misc}
\begin{tabular}{*3{ll}}
\X[\AMSangle]\angle  & \X\complement        & \X\measuredangle     \\
\X\backprime         & \X\diagdown          & \X\mho               \\
\X\Bbbk              & \X\diagup            & \X\nexists           \\
\X\bigstar           & \X\eth               & \X\sphericalangle    \\
\X\blacklozenge      & \X\Finv              & \X\square            \\
\X\blacksquare       & \X\Game              & \X\triangledown      \\
\X\blacktriangle     & \X\hbar              & \X\varnothing        \\
\X\blacktriangledown & \X\hslash            & \X\vartriangle       \\
\X\circledS          & \X\lozenge           \\
\end{tabular}
\end{symtable}


\begin{symtable}[AMS]{\AMS\ Commands Defined to Work in Both Math and Text Mode}
\label{ams-math-text}
\begin{tabular}{*2{ll@{\qquad}}ll}
\X\checkmark & \X\circledR & \X\maltese
\end{tabular}
\end{symtable}


\begin{symtable}[ST]{\ST\ Extension Characters}
\index{extensions}
\label{st-ext}
\begin{tabular}{*8l}
\X\Arrownot   &\X\Mapsfromchar &\X\Mapstochar \\
\X\arrownot   &\X\mapsfromchar
\end{tabular}
\end{symtable}


\begin{symtable}[WASY]{Other \WASY\ Math-Mode Symbols}
\idxboth{math-mode}{symbols}
\label{wasy-math}
\begin{tabular}{*4{ll}}
\X\apprge     & \X\Join       & \X\mho        & \X\sqsupset   \\
\X\apprle     & \X\leadsto    & \X\ocircle    & \X\unlhd      \\
\X\Box        & \X\lhd        & \X\rhd        & \X\unrhd      \\
\X\Diamond    & \X\LHD        & \X\RHD        & \X\wasypropto \\
\X\invneg     & \X\logof      & \X\sqsubset   \\
\end{tabular}
\end{symtable}


\begin{symtable}[TX]{Miscellaneous \TXPX\ Symbols}
\idxboth{miscellaneous}{symbols}
\index{diamonds (suit)}
\index{hearts (suit)}
\index{clubs (suit)}
\index{spades (suit)}
\label{txpx-misc}
\begin{tabular}{*4{ll}}
\X\Diamondblack & \X\lambdaslash  & \X\varclubsuit    & \X\varspadesuit \\
\X\Diamonddot   & \X\mathcent     & \X\vardiamondsuit \\
\X\lambdabar    & \X\mathsterling & \X\varheartsuit   \\
\end{tabular}
\end{symtable}


\begin{symtable}{\TC\ Text-Mode Math and Science Symbols}
\label{tc-math-science}
\begin{tabular}{*3{ll}}
\K\textcelsius         & \K\textminus       & \K\textsurd          \\
\K\textdegree          & \K\textmu          & \K\textthreequarters \\
\K\textdiv             & \K\textohm         & \K\textthreesuperior \\
\K\textdownarrow       & \K\textonehalf     & \K\texttimes         \\
\K\textfractionsolidus & \K\textonequarter  & \K\texttwosuperior   \\
\K\textleftarrow       & \K\textonesuperior & \K\textuparrow       \\
\K\textlnot            & \K\textpm          \\
\K\textmho             & \K\textrightarrow  \\
\end{tabular}
\end{symtable}


\begin{symtable}[MARV]{\MARV\ Math Symbols}
\index{digits}
\label{marv-math}
\begin{tabular}{*4{ll@{\qquad}}ll}
\K\MVZero  & \K\MVTwo   & \K\MVFour  & \K\MVSix   & \K\MVEight \\
\K\MVOne   & \K\MVThree & \K\MVFive  & \K\MVSeven & \K\MVNine  \\
\end{tabular}

\bigskip
\begin{tabular}{*3{ll@{\qquad}}ll}
\K\Anglesign       & \K\Squaredot       & \K\Vectorarrowhigh \\
\K\Corresponds     & \K\Vectorarrow     \\
\end{tabular}
\end{symtable}


\begin{symtable}[ASP]{\ASP\ Aspect Ratio Symbol}
\index{aspect ratio}
\label{aspect-ratio}
\begin{tabular}{ll}
\K\AR
\end{tabular}
\end{symtable}


\begin{symtable}[ULSY]{\ULSY\ Contradiction and Other Symbols}
\idxboth{contradiction}{symbols}
\label{ulsy}\medskip
\begin{tabular}{*6{ll}}
\K\blitza & \K\blitzb & \K\blitzc & \K\blitzd & \K\blitze & \K\odplus \\
\end{tabular}
\end{symtable}


\begin{symtable}{Math Alphabets}
\idxboth{math}{alphabets}
\index{blackboard bold}
\index{fraktur}
\index{script}
\label{alphabets}
\begin{tabular}{*3l}
        &               &Required package                    \\
\hline
\W\mathrm{ABCdef123}    & \textit{none}                      \\
\W\mathit{ABCdef123}    & \textit{none}                      \\
\W\mathnormal{ABCdef123}& \textit{none}                      \\
\Ww\CMcal\mathcal{ABC}  & \textit{none}                      \\

\ifx\mathscr\undefined\else
\W\mathscr{ABC}         &\pkgname{mathrsfs} \\
\fi

\ifEU
\W\mathcal{ABC}         &\pkgname{euscript}  with option: \optname{mathcal}  \\
\multicolumn{1}{r@{}}{\emph{or}}
        &\verb|\mathscr{ABC}|
                        &\pkgname{euscript}  with option: \optname{mathscr} \\
\fi

\ifx\mathpzc\undefined\else
\W\mathpzc{ABCdef123}   & \textit{none}; manually defined$^*$    \\
\fi

\ifx\mathbb\undefined\else
\W\mathbb{ABC}          &\pkgname{amsfonts}, \pkgname{amssymb},
                         \pkgname{txfonts}, or \pkgname{pxfonts} \\
\fi

\ifx\varmathbb\undefined\else
\W\varmathbb{ABC}       &\pkgname{txfonts} or \pkgname{pxfonts} \\
\fi

\ifx\BBmathbb\undefined\else
\Ww\BBmathbb\mathbb{ABCdef123}
                        &\pkgname{bbold} or \pkgname{mathbbol}$^\dag$  \\
\fi

\ifx\mathbbm\undefined\else
\W\mathbbm{ABCdef12}    &\pkgname{bbm}                         \\
\W\mathbbmss{ABCdef12}  &\pkgname{bbm}                         \\
\W\mathbbmtt{ABCdef12}  &\pkgname{bbm}                         \\
\fi

\ifx\mathds\undefined\else
\W\mathds{ABC1}         &\pkgname{dsfont}                      \\
\Ww\mathdsss\mathds{ABC1}
                        &\pkgname{dsfont} with option: \optname{sans} \\
\fi

\ifx\mathfrak\undefined\else
\W\mathfrak{ABCdef123}  &\pkgname{eufrak}                      \\
\fi

\ifx\textfrak\undefined\else
\W\textfrak{ABCdef123}  &\pkgname{yfonts}                      \\
\W\textswab{ABCdef123}  &\pkgname{yfonts}                      \\
\fi
\end{tabular}

\ifx\mathpzc\undefined\else
\bigskip
\begin{tablenote}[*]
  Put ``\verb|\DeclareMathAlphabet{\mathpzc}{OT1}{pzc}{m}{it}|'' in your
  document's preamble to make \verb|\mathpzc| typeset its argument in
  Zapf Chancery\index{Zapf Chancery}.
\end{tablenote}
\fi

\ifx\BBmathbb\undefined\else
\bigskip
\begin{tablenote}[\dag]
  The \pkgname{mathbbol} package defines some additional blackboard bold
  characters: parentheses, square brackets, angle brackets, and---if the
  \optname{bbgreekl} option is passed to
  \pkgname{matbbol}---Greek\index{Greek} letters.  For instance,
  ``$\BBmathbb{\char`<\char`[\char`(\char"0B\char"0C\char"0D\char`)\char`]\char`>}$''
  is produced by
  ``\cmd{\mathbb}\verb|{|\cmd{\Langle}\linebreak[1]\cmd{\Lbrack}\linebreak[1]%
  \cmd{\Lparen}\linebreak[1]\cmd{\bbalpha}\linebreak[1]%
  \cmd{\bbbeta}\linebreak[1]\cmd{\bbgamma}\linebreak[1]%
  \cmd{\Rparen}\linebreak[1]\cmd{\Rbrack}\linebreak[1]%
  \cmd{\Rangle}\verb|}|''.
\end{tablenote}
\fi

\end{symtable}



\section{Science and technology symbols}
\idxboth{scientific}{symbols}
\idxboth{technological}{symbols}

This section lists symbols that are employed in various branches of
science and engineering (and, because we were extremely liberal in our
classification, astrology, too).

\bigskip


\begin{symtable}[WASY]{\WASY\ Electrical and Physical Symbols}
\idxboth{electrical}{symbols}
\idxboth{physical}{symbols}
\label{wasy-electic}
\begin{tabular}{*{9}{ll@{\qquad}}ll}
\K\AC             & \K\VHF            & \K\photon         &
\K\HF             & \K\gluon          \\
\end{tabular}
\end{symtable}


\begin{symtable}[IFS]{\IFS\ Pulse Diagram Symbols}
\idxboth{pulse diagram}{symbols}
\idxboth{engineering}{symbols}
\label{pulse-diagram}
\begin{tabular}{*4{ll}}
\K\FallingEdge   & \K\LongPulseLow & \K\PulseLow    & \K\ShortPulseHigh \\
\K\LongPulseHigh & \K\PulseHigh    & \K\RaisingEdge & \K\ShortPulseLow  \\
\end{tabular}

\bigskip
\begin{tablenote}
  In addition, within
  \verb|\textifsym{|$\ldots$\verb|}|\indexcommand{\textifsym}, the
  following codes are valid:

  \begin{center}
  \begin{tabular}{*7{ll@{\qquad}}ll}
    \textifsym{l} & l &
    \textifsym{m} & m &
    \textifsym{h} & h &
    \textifsym{d} & d &
    \textifsym{<} & \textless &
    \textifsym{>} & \textgreater \\[4pt]

    \textifsym{L} & L &
    \textifsym{M} & M &
    \textifsym{H} & H &
    \textifsym{D} & D &
    \textifsym{<<} & \textless\textless &
    \textifsym{>>} & \textgreater\textgreater \\
  \end{tabular}
  \end{center}

  This enables one to write ``\verb|\textifsym{mm<DDD>mm}|'' to get
  ``\textifsym{mm<DDD>mm}'' or ``\verb+\textifsym{L|H|L|H|L}+'' to get
  ``\textifsym{L|H|L|H|L}''.

  Finally, \cmd{\textifsym} supports the display of
  segmented\idxboth{segmented}{digits} digits, as would appear on an
  LCD\index{LCD}: ``\verb|\textifsym{-123.456}|'' produces
  ``\textifsym{-123.456}''.  ``\verb|\textifsym{b}|'' outputs a blank
  with the same width as an ``\textifsym{8}''.
\end{tablenote}
\end{symtable}


\begin{symtable}[WASY]{\WASY\ Astronomical Symbols}
\idxboth{astronomical}{symbols}
\label{wasy-astro}
\begin{tabular}{*8l}
\K\ascnode        & \K\jupiter        & \K\newmoon        & \K\venus      \\
\K\astrosun       & \K\leftmoon       & \K\pluto          & \K\vernal     \\
\K\descnode       & \K\mars           & \K\rightmoon      \\
\K\earth          & \K\mercury        & \K\saturn         \\
\K\fullmoon       & \K\neptune        & \K\uranus         \\
\end{tabular}
\end{symtable}


\begin{symtable}[MARV]{\MARV\ Astronomical Symbols}
\idxboth{astronomical}{symbols}
\label{marv-astronomy}
\begin{tabular}{*5{ll}}
\K\Mercury & \K\Mars    & \K\Uranus  & \K\Sun     \\
\K\Venus   & \K\Jupiter & \K\Neptune & \K\Moon    \\
\K\Earth   & \K\Saturn  & \K\Pluto   \\
\end{tabular}
\end{symtable}


\begin{symtable}[WASY]{\WASY\ Astrological Symbols}
\idxboth{astrological}{symbols}
\idxboth{zodiacal}{symbols}
\label{wasy-astrology}
\begin{tabular}{*4{ll}}
\K\aries       & \K\cancer      & \K\libra       & \K\capricornus \\
\K\taurus      & \K\leo         & \K\scorpio     & \K\aquarius    \\
\K\gemini      & \K\virgo       & \K\sagittarius & \K\pisces      \\
\end{tabular}

\bigskip

\begin{tabular}{*2{ll}}
\K\conjunction & \K\opposition
\end{tabular}
\end{symtable}


\begin{symtable}[MARV]{\MARV\ Astrological Symbols}
\idxboth{astrological}{symbols}
\idxboth{zodiacal}{symbols}
\label{marv-astrology}
\begin{tabular}{*4{ll}}
\K\Aries       & \K\Cancer      & \K\Libra       & \K\Capricorn   \\
\K\Taurus      & \K\Leo         & \K\Scorpio     & \K\Aquarius    \\
\K\Gemini      & \K\Virgo       & \K\Sagittarius & \K\Pisces      \\
\end{tabular}

\bigskip
\begin{tablenote}
  Note that \cmd{\Aries}\,$\ldots$\,\linebreak[1]\cmd{\Pisces} can also be
  specified with
  \cmd{\Zodiac}\verb|{1}|\,$\ldots$\,\linebreak[1]\cmd{\Zodiac}\verb|{12}|.
\end{tablenote}
\end{symtable}


\begin{symtable}[WASY]{\WASY\ APL Symbols}
\index{APL>symbols}
\index{symbols>APL}
\label{wasy-APLsym}
\begin{tabular}{*6l}
\K\APLbox         & \K\APLinv           & \K\APLstar        \\
\K\APLcomment     & \K\APLleftarrowbox  & \K\APLup          \\
\K\APLdown        & \K\APLlog           & \K\APLuparrowbox  \\
\K\APLdownarrowbox & \K\APLminus        & \K\notbackslash   \\
\K\APLinput       & \K\APLrightarrowbox & \K\notslash       \\
\end{tabular}
\end{symtable}


\begin{symtable}[WASY]{\WASY\ APL Modifiers}
\index{APL>modifiers}
\label{wasy-APLmod}
\begin{tabular}{*6l}
\W\APLcirc{}        & \W\APLnot{}         & \W\APLvert{}        \\
\end{tabular}
\end{symtable}


\begin{symtable}[MARV]{\MARV\ Computer Hardware Symbols}
\idxboth{computer hardware}{symbols}
\label{marv-computer}
\begin{tabular}{*2{ll}ll}
\K\ComputerMouse   & \K\ParallelPort    & \K\SerialInterface \\
\K\Keyboard        & \K\Printer         & \K\SerialPort      \\
\end{tabular}
\end{symtable}


\begin{symtable}[ASCII]{ASCII Control Characters (IBM)}
\index{ASCII}\index{IBM}\index{control characters}\index{carriage return}
\label{ibm-ascii}
\begin{tabular}{*5{ll@{\hspace{3em}}}ll}
\Ka\SOH & \Ka\BEL & \Ka\CR  & \Ka\DCc & \Ka\EM  & \Ka\US        \\
\Ka\STX & \Ka\BS  & \Ka\SO  & \Ka\DCd & \Ka\SUB & \Ka\splitvert \\
\Ka\ETX & \Ka\HT  & \Ka\SI  & \Ka\NAK & \Ka\ESC & \Ka\DEL       \\
\Ka\EOT & \Ka\LF  & \Ka\DLE & \Ka\SYN & \Ka\FS  \\
\Ka\ENQ & \Ka\VT  & \Ka\DCa & \Ka\ETB & \Ka\GS  \\
\Ka\ACK & \Ka\FF  & \Ka\DCb & \Ka\CAN & \Ka\RS  \\
\end{tabular}

\bigskip

\begin{tablenote}
  \texttt{SOH}, \texttt{STX}, \texttt{ETX},~$\ldots$, \texttt{US} are
  the names of ASCII characters~1--31.  \texttt{DEL} is the name of
  ASCII character~127.  \cmd{\splitvert} doesn't correspond to a control
  character but is merely the ``$|$'' character shown IBM style.

  These characters require the \ASCII\ package and must be entered with
  the \texttt{ascii} font in effect, for example,
  ``\verb|{\ascii\STX}|''.  See the \ASCII\ package documentation for
  more information.
\end{tablenote}
\end{symtable}


\begin{symtable}[MARV]{\MARV\ Communication Symbols}
\idxboth{communication}{symbols}
\label{marv-comm}
\begin{tabular}{*4{ll}ll}
\K\Email      & \K\fax        & \K\Faxmachine & \K\Lightning  & \K\Pickup  \\
\K\Emailct    & \K\FAX        & \K\Letter     & \K\Mobilefone & \K\Telefon \\
\end{tabular}
\end{symtable}


\begin{symtable}[MARV]{\MARV\ Engineering Symbols}
\idxboth{engineering}{symbols}
\label{marv-engineering}
\begin{tabular}{*3{ll}ll}
\K\Beam         & \K\Force          & \K\Octosteel      & \K\RoundedTTsteel \\
\K\Bearing      & \K\Hexasteel      & \K\Rectpipe       & \K\Squarepipe     \\
\K\Circpipe     & \K\Lefttorque     & \K\Rectsteel      & \K\Squaresteel    \\
\K\Circsteel    & \K\Lineload       & \K\Righttorque    & \K\Tsteel         \\
\K\Fixedbearing & \K\Loosebearing   & \K\RoundedLsteel  & \K\TTsteel        \\
\K\Flatsteel    & \K\Lsteel         & \K\RoundedTsteel  \\
\end{tabular}
\end{symtable}


\begin{symtable}[MARV]{\MARV\ Biological Symbols}
\idxboth{biological}{symbols}
\label{marv-bio}
\begin{tabular}{*3{ll}ll}
\K\Female        & \K\FemaleMale    & \K\MALE          & \K\Neutral       \\
\K\FEMALE        & \K\Hermaphrodite & \K\Male          \\
\K\FemaleFemale  & \K\HERMAPHRODITE & \K\MaleMale      \\
\end{tabular}
\end{symtable}


\begin{symtable}[MARV]{\MARV\ Safety-Related Symbols}
\idxboth{safety-related}{symbols}
\label{marv-safety}
\begin{tabular}{*3{ll}ll}
\K\Biohazard     & \K\CEsign        & \K\Explosionsafe & \K\Radioactivity \\
\K\BSEfree       & \K\Estatically   & \K\Laserbeam     & \K\Stopsign      \\
\end{tabular}
\end{symtable}


\section{Dingbats}
\index{dingbats}

Dingbats are symbols such as stars, arrows, and geometric shapes.
They are commonly used as bullets in itemized lists or, more
generally, as a means to draw attention to the text that follows.

The \PI\ dingbat package warrants special mention.  Among other
capabilities, \PI\ provides a \latex\ interface to the
PostScript\index{PostScript} Zapf\index{Zapf Dingbats} Dingbats font.
However, rather than name each of the dingbats individually, \PI\
merely provides a single \cmd{\ding} command, which outputs the
character that lies at a given position in the font.  The consequence
is that the \PI\ symbols can't be listed by name in this document's
index, so be mindful of that fact when searching for a particular
symbol.

\bigskip


\begin{symtable}[DING]{\DING\ Arrows}
\label{bbding-arrows}
\begin{tabular}{*3{ll}}
\K\ArrowBoldDownRight    & \K\ArrowBoldRightShort  & \K\ArrowBoldUpRight \\
\K\ArrowBoldRightCircled & \K\ArrowBoldRightStrobe \\
\end{tabular}
\end{symtable}


\begin{symtable}[PI]{\PI\ Arrows}
\index{arrows}
\label{pi-arrows}
\begin{tabular}{*5{ll}}
\Tp{212} & \Tp{221} & \Tp{230} & \Tp{239} & \Tp{249} \\
\Tp{213} & \Tp{222} & \Tp{231} & \Tp{241} & \Tp{250} \\
\Tp{214} & \Tp{223} & \Tp{232} & \Tp{242} & \Tp{251} \\
\Tp{215} & \Tp{224} & \Tp{233} & \Tp{243} & \Tp{252} \\
\Tp{216} & \Tp{225} & \Tp{234} & \Tp{244} & \Tp{253} \\
\Tp{217} & \Tp{226} & \Tp{235} & \Tp{245} & \Tp{254} \\
\Tp{218} & \Tp{227} & \Tp{236} & \Tp{246} \\
\Tp{219} & \Tp{228} & \Tp{237} & \Tp{247} \\
\Tp{220} & \Tp{229} & \Tp{238} & \Tp{248} \\
\end{tabular}
\end{symtable}


\begin{symtable}[MARV]{\MARV\ Scissors}
\index{scissors}
\label{marv-scissors}
\begin{tabular}{*3{ll}}
\K\Cutleft       & \K\Cutright      & \K\Leftscissors  \\
\K\Cutline       & \K\Kutline       & \K\Rightscissors \\
\end{tabular}
\end{symtable}


\begin{symtable}[DING]{\DING\ Scissors}
\index{scissors}
\label{scissors}
\begin{tabular}{*2{ll}}
\K\ScissorHollowLeft        & \K\ScissorLeftBrokenTop     \\
\K\ScissorHollowRight       & \K\ScissorRight             \\
\K\ScissorLeft              & \K\ScissorRightBrokenBottom \\
\K\ScissorLeftBrokenBottom  & \K\ScissorRightBrokenTop    \\
\end{tabular}
\end{symtable}


\begin{symtable}[PI]{\PI\ Scissors}
\index{scissors}
\label{pi-scissors}
\begin{tabular}{*4{ll}}
\Tp{33} & \Tp{34} & \Tp{35} & \Tp{36} \\
\end{tabular}
\end{symtable}


\begin{symtable}[ARK]{\ARK\ Pencils}
\index{pencils}
\vspace{1ex}
\begin{tabular}{*2{ll}}
\K\largepencil & \K\smallpencil \\
\end{tabular}
\end{symtable}


\begin{symtable}[DING]{\DING\ Pencils and Nibs}
\index{pencils}
\index{nibs}
\label{pencils-nibs}
\begin{tabular}{*3{ll}}
\K\NibLeft         & \K\PencilLeft      & \K\PencilRightDown \\
\K\NibRight        & \K\PencilLeftDown  & \K\PencilRightUp   \\
\K\NibSolidLeft    & \K\PencilLeftUp    \\
\K\NibSolidRight   & \K\PencilRight     \\
\end{tabular}
\end{symtable}


\begin{symtable}[PI]{\PI\ Pencils and Nibs}
\index{pencils}
\index{nibs}
\label{pi-pencils}
\begin{tabular}{*5{ll}}
\Tp{46} & \Tp{47} & \Tp{48} & \Tp{49} & \Tp{50} \\
\end{tabular}
\end{symtable}


\begin{symtable}[ARK]{\ARK\ Hands}
\index{hands}
\label{ark-hands}
\renewcommand{\arraystretch}{1.25}
\begin{tabular}{*3{ll}}
\K\leftpointright  & \K\rightpointleft  & \K\rightpointright \\
\K\leftthumbsdown  & \K\rightthumbsdown \\
\K\leftthumbsup    & \K\rightthumbsup   \\
\end{tabular}
\end{symtable}


\begin{symtable}[DING]{\DING\ Hands}
\index{hands}
\label{hands}
\begin{tabular}{*3{ll}}
\K\HandCuffLeft    & \K\HandCuffRightUp & \K\HandPencilLeft  \\
\K\HandCuffLeftUp  & \K\HandLeft        & \K\HandRight       \\
\K\HandCuffRight   & \K\HandLeftUp      & \K\HandRightUp     \\
\end{tabular}
\end{symtable}


\begin{symtable}[PI]{\PI\ Hands}
\index{hands}
\label{pi-hands}
\begin{tabular}{*4{ll}}
\Tp{42} & \Tp{43} & \Tp{44} & \Tp{45} \\
\end{tabular}
\end{symtable}


\begin{symtable}[DING]{\DING\ Crosses and Plusses}
\index{crosses}
\index{plusses}
\index{crucifixes}
\label{crosses-plusses}
\begin{tabular}{*3{ll}}
\K[\dingCross]\Cross  & \K\CrossOpenShadow    & \K\PlusOutline        \\
\K\CrossBoldOutline   & \K\CrossOutline       & \K\PlusThinCenterOpen \\
\K\CrossClowerTips    & \K\Plus               \\
\K\CrossMaltese       & \K\PlusCenterOpen     \\
\end{tabular}
\end{symtable}


\begin{symtable}[PI]{\PI\ Crosses and Plusses}
\index{crosses}
\index{plusses}
\index{crucifixes}
\label{pi-crosses-plusses}
\begin{tabular}{*4{ll}}
\Tp{57} & \Tp{59} & \Tp{61} & \Tp{63} \\
\Tp{58} & \Tp{60} & \Tp{62} & \Tp{64} \\
\end{tabular}
\end{symtable}


\begin{symtable}[DING]{\DING\ Xs and Check Marks}
\index{check marks}
\index{Xs}
\label{ding-check-marks}
\begin{tabular}{*3{ll}}
\K\Checkmark     & \K\XSolid        & \K\XSolidBrush   \\
\K\CheckmarkBold & \K\XSolidBold    \\
\end{tabular}
\end{symtable}


\begin{symtable}[PI]{\PI\ Xs and Check Marks}
\index{check marks}
\index{Xs}
\label{pi-check-marks}
\begin{tabular}{*3{ll}}
\Tp{51} & \Tp{53} & \Tp{55} \\
\Tp{52} & \Tp{54} & \Tp{56} \\
\end{tabular}
\end{symtable}


\begin{symtable}[WASY]{\WASY\ Xs and Check Marks}
\index{check marks}
\index{Xs}
\label{wasy-check-marks}
\begin{tabular}{*6l}
\K\CheckedBox & \K\Square & \K\XBox
\end{tabular}
\end{symtable}


\begin{symtable}[PI]{\PI\ Circled Numbers}
\index{circled numbers}
\index{digits>circled}
\label{circled-numbers}
\begin{tabular}{*4{ll}}
\Tp{172} & \Tp{182} & \Tp{192} & \Tp{202} \\
\Tp{173} & \Tp{183} & \Tp{193} & \Tp{203} \\
\Tp{174} & \Tp{184} & \Tp{194} & \Tp{204} \\
\Tp{175} & \Tp{185} & \Tp{195} & \Tp{205} \\
\Tp{176} & \Tp{186} & \Tp{196} & \Tp{206} \\
\Tp{177} & \Tp{187} & \Tp{197} & \Tp{207} \\
\Tp{178} & \Tp{188} & \Tp{198} & \Tp{208} \\
\Tp{179} & \Tp{189} & \Tp{199} & \Tp{209} \\
\Tp{180} & \Tp{190} & \Tp{200} & \Tp{210} \\
\Tp{181} & \Tp{191} & \Tp{201} & \Tp{211} \\
\end{tabular}
\end{symtable}


\begin{symtable}[WASY]{\WASY\ Stars}
\index{stars}
\index{Jewish star}\index{Star of David}
\label{wasy-stars}
\begin{tabular}{*6l}
\K\davidsstar & \K\hexstar & \K\varhexstar
\end{tabular}
\end{symtable}


\begin{symtable}[DING]{\DING\ Stars, Flowers, and Similar Shapes}
\index{asterisks}
\index{clovers}
\index{flowers}
\index{sparkles}
\index{snowflakes}
\index{stars}
\index{Jewish star}\index{Star of David}
\label{star-like}
\begin{tabular}{*3{ll}}
\K\Asterisk                & \K\FiveFlowerPetal      & \K\JackStar                  \\
\K\AsteriskBold            & \K\FiveStar             & \K\JackStarBold              \\
\K\AsteriskCenterOpen      & \K\FiveStarCenterOpen   & \K\SixFlowerAlternate        \\
\K\AsteriskRoundedEnds     & \K\FiveStarConvex       & \K\SixFlowerAltPetal         \\
\K\AsteriskThin            & \K\FiveStarLines        & \K\SixFlowerOpenCenter       \\
\K\AsteriskThinCenterOpen  & \K\FiveStarOpen         & \K\SixFlowerPetalDotted      \\
\K\DavidStar               & \K\FiveStarOpenCircled  & \K\SixFlowerPetalRemoved     \\
\K\DavidStarSolid          & \K\FiveStarOpenDotted   & \K\SixFlowerRemovedOpenPetal \\
\K\EightAsterisk           & \K\FiveStarOutline      & \K\SixStar                   \\
\K\EightFlowerPetal        & \K\FiveStarOutlineHeavy & \K\SixteenStarLight          \\
\K\EightFlowerPetalRemoved & \K\FiveStarShadow       & \K\Snowflake                 \\
\K\EightStar               & \K\FourAsterisk         & \K\SnowflakeChevron          \\
\K\EightStarBold           & \K\FourClowerOpen       & \K\SnowflakeChevronBold      \\
\K\EightStarConvex         & \K\FourClowerSolid      & \K\Sparkle                   \\
\K\EightStarTaper          & \K\FourStar             & \K\SparkleBold               \\
\K\FiveFlowerOpen          & \K\FourStarOpen         & \K\TwelweStar                \\
\end{tabular}
\end{symtable}


\begin{symtable}[PI]{\PI\ Stars, Flowers, and Similar Shapes}
\index{asterisks}
\index{clovers}
\index{flowers}
\index{sparkles}
\index{snowflakes}
\index{stars}
\label{pi-star-like}
\begin{tabular}{*5{ll}}
\Tp{65} & \Tp{74} & \Tp{83} & \Tp{92} & \Tp{101} \\
\Tp{66} & \Tp{75} & \Tp{84} & \Tp{93} & \Tp{102} \\
\Tp{67} & \Tp{76} & \Tp{85} & \Tp{94} & \Tp{103} \\
\Tp{68} & \Tp{77} & \Tp{86} & \Tp{95} & \Tp{104} \\
\Tp{69} & \Tp{78} & \Tp{87} & \Tp{96} & \Tp{105} \\
\Tp{70} & \Tp{79} & \Tp{88} & \Tp{97} & \Tp{106} \\
\Tp{71} & \Tp{80} & \Tp{89} & \Tp{98} & \Tp{107} \\
\Tp{72} & \Tp{81} & \Tp{90} & \Tp{99} \\
\Tp{73} & \Tp{82} & \Tp{91} & \Tp{100} \\
\end{tabular}
\end{symtable}


\begin{symtable}[WASY]{\WASY\ Geometric Shapes}
\index{polygons}
\index{geometric shapes}
\label{wasy-geometrical}
\begin{tabular}{*8l}
\K\hexagon & \K\octagon & \K\pentagon & \K\varhexagon
\end{tabular}
\end{symtable}


\begin{symtable}[IFS]{\IFS\ Geometric Shapes}
\index{circles}
\index{diamonds}
\index{geometric shapes}
\index{squares}
\index{triangles}
\label{ifs-geometrical}
\begin{tabular}{*3{ll}}
\K\BigCircle             & \K\FilledBigTriangleRight   & \K\SmallCircle        \\
\K\BigCross              & \K\FilledBigTriangleUp      & \K\SmallCross         \\
\K\BigDiamondshape       & \K\FilledCircle             & \K\SmallDiamondshape  \\
\K\BigHBar               & \K\FilledDiamondShadowA     & \K\SmallHBar          \\
\K\BigLowerDiamond       & \K\FilledDiamondShadowC     & \K\SmallLowerDiamond  \\
\K\BigRightDiamond       & \K\FilledDiamondshape       & \K\SmallRightDiamond  \\
\K\BigSquare             & \K\FilledSmallCircle        & \K\SmallSquare        \\
\K\BigTriangleDown       & \K\FilledSmallDiamondshape  & \K\SmallTriangleDown  \\
\K\BigTriangleLeft       & \K\FilledSmallSquare        & \K\SmallTriangleLeft  \\
\K\BigTriangleRight      & \K\FilledSmallTriangleDown  & \K\SmallTriangleRight \\
\K\BigTriangleUp         & \K\FilledSmallTriangleLeft  & \K\SmallTriangleUp    \\
\K\BigVBar               & \K\FilledSmallTriangleRight & \K\SmallVBar          \\
\K[\ifsCircle]\Circle    & \K\FilledSmallTriangleUp    & \K\SpinDown           \\
\K[\ifsCross]\Cross      & \K\FilledSquare             & \K\SpinUp             \\
\K\DiamondShadowA        & \K\FilledSquareShadowA      & \K[\ifsSquare]\Square \\
\K\DiamondShadowB        & \K\FilledSquareShadowC      & \K\SquareShadowA      \\
\K\DiamondShadowC        & \K\FilledTriangleDown       & \K\SquareShadowB      \\
\K\Diamondshape          & \K\FilledTriangleLeft       & \K\SquareShadowC      \\
\K\FilledBigCircle       & \K\FilledTriangleRight      & \K[\ifsTriangleDown]\TriangleDown \\
\K\FilledBigDiamondshape & \K\FilledTriangleUp         & \K\TriangleLeft       \\
\K\FilledBigSquare       & \K\HBar                     & \K\TriangleRight      \\
\K\FilledBigTriangleDown & \K\LowerDiamond             & \K[\ifsTriangleUp]\TriangleUp \\
\K\FilledBigTriangleLeft & \K\RightDiamond             & \K\VBar               \\
\end{tabular}

\bigskip
\begin{tablenote}
  \begin{morespacing}{1pt}
    The \IFS\ documentation points out that one can use \cmd{\rlap} to
    combine some of the above into useful, new symbols.  For example,
    \cmd{\BigCircle} and \cmd{\FilledSmallCircle} combine to give
    ``\,\rlap\FilledSmallCircle\BigCircle\,''.  Likewise, \cmd{\Square}
    and \cmd{\Cross} combine to give ``\rlap\ifsCross\ifsSquare''.  See
    Section~\ref{combining-symbols} for more information about
    constructing new symbols out of existing symbols.
  \end{morespacing}
\end{tablenote}
\end{symtable}


\begin{symtable}[DING]{\DING\ Geometric Shapes}
\index{circles}
\index{diamonds}
\index{ellipses}
\index{geometric shapes}
\index{rectangles}
\index{squares}
\index{triangles}
\label{ding-geometrical}
\begin{tabular}{*3{ll}}
\K\CircleShadow    & \K\Rectangle                   & \K\SquareShadowTopLeft     \\
\K\CircleSolid     & \K\RectangleBold               & \K\SquareShadowTopRight    \\
\K\DiamondSolid    & \K\RectangleThin               & \K\SquareSolid             \\
\K\Ellipse         & \K[\dingSquare]\Square         & \K\TriangleDown            \\
\K\EllipseShadow   & \K\SquareCastShadowBottomRight & \K\TriangleUp              \\
\K\EllipseSolid    & \K\SquareCastShadowTopLeft     \\
\K\HalfCircleLeft  & \K\SquareCastShadowTopRight    \\
\K\HalfCircleRight & \K\SquareShadowBottomRight     \\
\end{tabular}
\end{symtable}


\begin{symtable}[PI]{\PI\ Geometric Shapes}
\index{circles}
\index{diamonds}
\index{geometric shapes}
\index{rectangles}
\index{squares}
\index{triangles}
\label{pi-geometrical}
\begin{tabular}{*5{ll}}
\Tp{108} & \Tp{111} & \Tp{114} & \Tp{117} & \Tp{121} \\
\Tp{109} & \Tp{112} & \Tp{115} & \Tp{119} & \Tp{122} \\
\Tp{110} & \Tp{113} & \Tp{116} & \Tp{120} \\
\end{tabular}
\end{symtable}


\begin{symtable}[MAN]{\MAN\ Dangerous Bend Symbols}
\idxboth{dangerous bend}{symbols}
\index{symbols>Knuth's}
\index{Knuth, Donald E.>symbols by}
\idxTBsyms
\label{dangerous-bend}
\begin{tabular}{*3{ll}}
\K\dbend              & \K\lhdbend            & \K\reversedvideodbend \\
\end{tabular}

\bigskip
\begin{tablenote}
   Note that these symbols descend far beneath the baseline.  \MAN\ also
   defines non-descending versions, which it calls, correspondingly,
   \cmd{\textdbend}, \cmd{\textlhdbend}, and
   \cmd{\textreversedvideodbend}.
\end{tablenote}
\end{symtable}


\begin{symtable}[MARV]{\MARV\ Information Symbols}
\idxboth{information}{symbols}
\index{check marks}
\index{Xs}
\label{marv-info}
\begin{tabular}{*3{ll}ll}
\K\Bicycle      & \K\Football     & \K\Pointinghand \\
\K\Checkedbox   & \K\Gentsroom    & \K\Wheelchair   \\
\K\Clocklogo    & \K\Industry     & \K\Writinghand  \\
\K\Coffeecup    & \K\Info         \\
\K\Crossedbox   & \K\Ladiesroom   \\
\end{tabular}
\end{symtable}


\begin{symtable}[ARK]{Miscellaneous \ARK\ Dingbats}
\idxboth{miscellaneous}{symbols}
\index{check marks}
\label{ark-misc}
\begin{tabular}{*3{ll}}
\K\anchor         & \K\eye                     & \K\Sborder        \\
\K\carriagereturn & \K\filledsquarewithdots    & \K\squarewithdots \\
\K[\ARKcheckmark]\checkmark & \K\satellitedish & \K\Zborder        \\
\end{tabular}
\end{symtable}


\begin{symtable}[DING]{Miscellaneous \DING\ Dingbats}
\idxboth{miscellaneous}{symbols}
\label{bbding-misc}
\begin{tabular}{*4{ll}}
\K\Envelope             & \K\Peace & \K\PhoneHandset & \K\SunshineOpenCircled \\
\K\OrnamentDiamondSolid & \K\Phone & \K\Plane        & \K\Tape                \\
\end{tabular}
\end{symtable}


\begin{symtable}[PI]{Miscellaneous \PI\ Dingbats}
\idxboth{miscellaneous}{symbols}
\index{diamonds (suit)}
\index{hearts (suit)}
\index{clubs (suit)}
\index{spades (suit)}
\label{pi-misc}
\begin{tabular}{*5{ll}}
\Tp{37} & \Tp{40}  & \Tp{164} & \Tp{167} & \Tp{171} \\
\Tp{38} & \Tp{41}  & \Tp{165} & \Tp{168} & \Tp{169} \\
\Tp{39} & \Tp{118} & \Tp{166} & \Tp{170} \\
\end{tabular}
\end{symtable}



\section{Other symbols}
\idxboth{miscellaneous}{symbols}

The following are all the symbols that didn't fit neatly or
unambiguously into any of the previous sections.
\ifcomplete
(Do weather symbols belong under ``Science and technology''?  Should
dice be considered ``mathematics''?)  While some of the tables contain
clearly related groups of symbols (e.g., musical notes), others
represent motley assortments of whatever the font designer felt like
drawing.
\fi

\bigskip


\begin{symtable}{\TC\ Genealogical Symbols}
\idxboth{genealogical}{symbols}
\label{genealogical}
\begin{tabular}{*3{ll}}
\K\textborn     & \K\textdivorced & \K\textmarried  \\
\K\textdied     & \K\textleaf     \\
\end{tabular}
\end{symtable}


\begin{symtable}[WASY]{\WASY\ General Symbols}
\index{symbols>general}
\label{wasy-general}
\begin{tabular}{*4{ll}}
\K\agemO         & \K\clock         & \K\LEFTarrow     & \K\smiley        \\
\K\ataribox      & \K\currency      & \K\lightning     & \K\sun           \\
\K\bell          & \K\diameter      & \K\male          & \K\UParrow       \\
\K\blacksmiley   & \K\DOWNarrow     & \K\permil        & \K\varangle      \\
\K\Bowtie        & \K\female        & \K\phone         & \K\wasylozenge   \\
\K\brokenvert    & \K\frownie       & \K\pointer       & \K\wasytherefore \\
\K\cent          & \K\invdiameter   & \K\recorder      \\
\K\checked       & \K\kreuz         & \K\RIGHTarrow    \\
\end{tabular}
\end{symtable}


\begin{symtable}[WASY]{\WASY\ Musical Notes}
\index{musical notes}
\label{wasy-music}
\begin{tabular}{*{10}l}
\K\eighthnote     & \K\halfnote       & \K\twonotes       &
\K\fullnote       & \K\quarternote    \\
\end{tabular}

\bigskip
\begin{tablenote}
  See also \cmd{\flat}, \cmd{\sharp}, and \cmd{\natural} (Table~\ref{ord}).
\end{tablenote}
\end{symtable}


\begin{symtable}[WASY]{\WASY\ Circles}
\index{circles}
\label{wasy-circles}
\begin{tabular}{*8l}
\K\CIRCLE         & \K\LEFTcircle     & \K\RIGHTcircle    & \K\rightturn      \\
\K\Circle         & \K\Leftcircle     & \K\Rightcircle    \\
\K\LEFTCIRCLE     & \K\RIGHTCIRCLE    & \K\leftturn       \\
\end{tabular}
\end{symtable}


\begin{symtable}[MAN]{Miscellaneous \MAN\ Symbols}
\index{symbols>Knuth's}
\index{Knuth, Donald E.>symbols by}
\index{symbols>Metafontbook=\MF{}book}\index{Metafontbook symbols=\MF{}book symbols}
\idxTBsyms
\label{knuth}
\begin{tabular}{*2{ll}}
\K\manboldkidney           & \K\manpenkidney            \\
\K\manconcentriccircles    & \K\manquadrifolium         \\
\K\manconcentricdiamond    & \K\manquartercircle        \\
\K\mancone                 & \K\manrotatedquadrifolium  \\
\K\mancube                 & \K\manrotatedquartercircle \\
\K\manerrarrow             & \K\manstar                 \\
\K\manfilledquartercircle  & \K\mantiltpennib           \\
\K\manhpennib              & \K\mantriangledown         \\
\K\manimpossiblecube       & \K\mantriangleright        \\
\K\mankidney               & \K\mantriangleup           \\
\K\manlhpenkidney          & \K\manvpennib              \\
\end{tabular}
\end{symtable}


\begin{symtable}[MARV]{\MARV\ Navigation Symbols}
\idxboth{navigation}{symbols}
\label{marv-navigation}
\begin{tabular}{*3{ll}ll}
\K\Forward        & \K\MoveDown  & \K\RewindToIndex  & \K\ToTop \\
\K\ForwardToEnd   & \K\MoveUp    & \K\RewindToStart  \\
\K\ForwardToIndex & \K\Rewind    & \K\ToBottom       \\
\end{tabular}
\end{symtable}


\begin{symtable}[MARV]{\MARV\ Laundry Symbols}
\idxboth{laundry}{symbols}
\label{marv-laundry}
\begin{tabular}{*3{ll}}
\K\AtForty            & \K\Handwash           & \K\ShortNinetyFive    \\
\K\AtNinetyFive       & \K\IroningI           & \K\ShortSixty         \\
\K\AtSixty            & \K\IroningII          & \K\ShortThirty        \\
\K\Bleech             & \K\IroningIII         & \K\SpecialForty       \\
\K\CleaningA          & \K\NoBleech           & \K\Tumbler            \\
\K\CleaningF          & \K\NoChemicalCleaning & \K\WashCotton         \\
\K\CleaningFF         & \K\NoIroning          & \K\WashSynthetics     \\
\K\CleaningP          & \K\NoTumbler          & \K\WashWool           \\
\K\CleaningPP         & \K\ShortFifty         \\
\K\Dontwash           & \K\ShortForty         \\
\end{tabular}
\end{symtable}


\begin{symtable}[MARV]{Other \MARV\ Symbols}
\idxboth{miscellaneous}{symbols}
\index{crosses}\index{crucifixes}
\label{marv-misc}
\begin{tabular}{*4{ll}}
\K\Ankh        & \K\Cross        & \K\Heart       & \K\Smiley      \\
\K\Bat         & \K\FHBOlogo     & \K\MartinVogel & \K\Womanface   \\
\K\Bouquet     & \K\FHBOLOGO     & \K\Mundus      & \K\Yinyang     \\
\K\Celtcross   & \K\Frowny       & \K\MVAt        \\
\K\CircledA    & \K\FullFHBO     & \K[\marvRightarrow]\Rightarrow$^*$ \\
\end{tabular}

\bigskip
\begin{tablenote}[*]
  Standard \latexE{} defines \cmd{\Rightarrow} to display
  ``$\Rightarrow$'', while \MARV\ redefines it to display
  ``\marvRightarrow'' (or ``$\marvRightarrow$'' in math mode).  This
  conflict can be problematic for math symbols defined in terms of
  \cmd{\Rightarrow}, such as \cmd{\Longleftrightarrow}, which ends up
  looking like ``$\Leftarrow\joinrel\marvRightarrow$''.
\end{tablenote}
\end{symtable}


\begin{symtable}[IFS]{\IFS\ Weather Symbols}
\idxboth{weather}{symbols}
\label{weather}
\begin{tabular}{*4{ll}}
\K\Blitz           & \K\FilledWeakRainCloud & \K\Rain      & \K\ThinFog       \\
\K\Cloud           & \K\Fog                 & \K\RainCloud & \K\WeakRain      \\
\K\FilledCloud     & \K\Graupel             & \K\Snow      & \K\WeakRainCloud \\
\K\FilledRainCloud & \K\Hagel               & \K\SnowCloud \\
\K\FilledSnowCloud & \K\HalfSun             & \K[\ifsSun]\Sun \\
\K\FilledSunCloud  & \K\NoSun               & \K\SunCloud  \\
\end{tabular}

\bigskip
\begin{tablenote}
  \begin{morespacing}{\jot}
    In addition,
    \verb|\Thermo{0}|$\ldots$\verb|\Thermo{6}|\indexcommand{\Thermo}
    produce thermometers that are between 0/6 and 6/6~full of
    mercury:\quad \mbox{\Thermo{0}~~\Thermo{1}~~\Thermo{2}~~\Thermo{3}~~%
    \Thermo{4}~~\Thermo{5}~~\Thermo{6}}
  \end{morespacing}

  % For portability, we don't bother with graphicx.
  \newcommand{\rotatebox}[2]{#2}

  \begin{morespacing}{1pt}

    Similarly,
    \cmd{\wind}\verb|{|\meta{sun}\verb|}{|\meta{angle}\verb|}{|\meta{strength}\verb|}|
    will draw wind symbols with a given amount of sun~(0--4), a given
    angle (in degrees), and a given strength in km/h~(0--100).  For
    example, \verb|\wind{0}{0}{0}| produces ``\,\wind{0}{0}{0}\unskip'',
    \verb|\wind{2}{0}{0}| produces ``\,\wind{2}{0}{0}\unskip'', and
    \verb|\wind{4}{0}{100}| produces ``\,\wind{4}{0}{100}\unskip''.

  \end{morespacing}
\end{tablenote}
\end{symtable}


\begin{symtable}[IFS]{\IFS\ Alpine Symbols}
\idxboth{alpine}{symbols}
\label{alpine}
\begin{tabular}{*4{ll}}
\K\FilledHut     & \K\Joch       & \K\Tent           & \K\Vermessung     \\
\K\Flag          & \K\Mountain   & \K\VarFlag        & \K\Village        \\
\K\HalfFilledHut & \K\StoneMan   & \K\VarIceMountain \\
\K\Hut           & \K\Summit     & \K\VarMountain    \\
\K\IceMountain   & \K\SummitSign & \K\VarSummit      \\
\end{tabular}
\end{symtable}


\begin{symtable}[IFS]{\IFS\ Clocks}
\idxboth{clock}{symbols}
\index{time of day}
\label{clocks}
\begin{tabular}{*4{ll}}
\K\Interval     & \K\StopWatchStart & \K\VarClock       & \K\Wecker \\
\K\StopWatchEnd & \K\Taschenuhr     & \K\VarTaschenuhr  \\
\end{tabular}

\bigskip
\begin{tablenote}
  \IFS\ also exports a \cmd{\showclock} macro.
  \verb|\showclock{|\meta{hours}\verb|}{|\meta{minutes}\verb|}| outputs
  a clock displaying the corresponding time.  For instance,
  ``\verb|\showclock{5}{40}|'' produces ``\showclock{5}{40}''.
  \meta{hours} must be an integer from 0 to~11, and \meta{minutes} must
  be an integer multiple of~5 from 0 to~55.
\end{tablenote}
\end{symtable}


\begin{symtable}[IFS]{Other \IFS\ Symbols}
\idxboth{miscellaneous}{symbols}
\index{tally markers}
\index{dice}
\label{ifs-misc}
\begin{tabular}{*3{ll}}
\K\FilledSectioningDiamond & \K[\ifsLetter]\Letter
                                               & \K\Radiation         \\
\K\Fire                    & \K\PaperLandscape & \K\SectioningDiamond \\
\K\Irritant                & \K\PaperPortrait  & \K\Telephone         \\[2ex]

\K\StrokeOne               & \K\StrokeThree    & \K\StrokeFive        \\
\K\StrokeTwo               & \K\StrokeFour  \\
\end{tabular}

\bigskip
\begin{tablenote}
  \begin{morespacing}{\jot}
    In addition,
    \verb|\Cube{1}|$\ldots$\verb|\Cube{6}|\indexcommand{\Cube} produce
    dice with the corresponding number of spots:\quad
    \mbox{\Cube{1}~~\Cube{2}~~\Cube{3}~~\Cube{4}~~\Cube{5}~~\Cube{6}}
  \end{morespacing}
\end{tablenote}
\end{symtable}


\section{Additional Information}
\label{addl-info}

Unlike the previous sections of this document, Section~\ref{addl-info}
does not contain new symbol tables.  Rather, it provides additional
help in using the \doctitle.  First, it draws attention to symbol
names used by multiple packages.  Then, it provides some guidelines
for finding symbols and gives some examples regarding how to construct
missing symbols out of existing ones.  Next, it comments on the
spacing surrounding symbols in math mode.  After that, it presents an
ASCII and Latin~1 quick-reference guide, showing how to enter all of
the standard ASCII/Latin~1 symbols in \latex{}.  And finally, it lists
some statistics about this document itself.

\subsection{Symbol Name Clashes}

% Rather than create a rat's nest of \if statements, we keep the table
% whole and have each symbol conditionally appear.
\makeatletter
\newcommand{\trysym}[1]{\@ifundefined{#1}{\mbox{\tiny N/A}}{\csname#1\endcsname}}
\makeatother

Unfortunately, a number of symbol names are not unique; they appear in
more than one package.  Depending on how the symbols are defined in
each package, \latex{} will either output an error message or replace
an earlier-defined symbol with a later-defined symbol.
Table~\ref{name-clashes} lists the name clashes that appear in this
document.  The symbol ``\trysym{NONEXISTENT}'' is used to indicate
that the corresponding package was not available when
\texttt{\jobname.tex} was compiled.

\begin{nonsymtable}{Symbol Name Clashes}
\label{name-clashes}
\begin{tabular}{@{}lp{0.3em}ccccccccc@{}} \toprule
  Symbol && \latexE & AMS & \ST & \WASY & \MARV & \DING & \IFS & \ARK & \WIPA \\
  \cmidrule(r){1-1}\cmidrule(l){3-11}
  %
  \cmd{\angle} &&
    $\angle$ & $\trysym{AMSangle}$ \\
  \cmd{\baro} &&
    & & $\trysym{baro}$ & & & & & & \trysym{WSUbaro} \\
  \cmd{\bigtriangledown} &&
    $\bigtriangledown$ & & $\trysym{STbigtriangledown}$ \\
  \cmd{\bigtriangleup} &&
    $\bigtriangleup$ & & $\trysym{STbigtriangleup}$ \\
  \cmd{\checkmark} &&
    & \trysym{checkmark} & & & & & & \trysym{ARKcheckmark} \\
  \cmd{\Circle} &&
    & & & \trysym{Circle} & & & \trysym{ifsCircle} \\
  \cmd{\Cross} &&
    & & & & \trysym{Cross} & \trysym{dingCross} & \trysym{ifsCross} \\
  \cmd{\Letter} &&
    & & & & \trysym{Letter} & & \trysym{ifsLetter} \\
  \cmd{\lightning} &&
    & & $\trysym{STlightning}$ & \trysym{WASYlightning} \\
  \cmd{\Rightarrow} &&
    $\Rightarrow$ & & & & \trysym{marvRightarrow} \\
  \cmd{\rightleftharpoons} &&
    $\rightleftharpoons$ & $\trysym{AMSrightleftharpoons}$ \\
  \cmd{\Square} &&
    & & & \trysym{Square} & & \trysym{dingSquare} & \trysym{ifsSquare} \\
  \cmd{\Sun} &&
    & & & & \trysym{Sun} & & \trysym{ifsSun} \\
  \cmd{\TriangleDown} &&
    & & & & & \trysym{TriangleDown} & \trysym{ifsTriangleDown} \\
  \cmd{\TriangleUp} &&
    & & & & & \trysym{TriangleUp} & \trysym{ifsTriangleUp} \\
  \bottomrule
\end{tabular}
\end{nonsymtable}


Using multiple symbols with the same name in the same document---or
even merely loading conflicting symbol packages---can be tricky, but,
as evidenced by the existence of Table~\ref{name-clashes}, not
impossible.  The general procedure is to load the first package,
rename the conflicting symbols, and then load the second package.
Examine the \latex{} source for this document---especially the
\cmd{\savesymbol} and \cmd{\restoresymbol} macros and their subsequent
usage---to see one possible way to handle symbol conflicts.

\ifTX

\TX\ and \PX\ redefine a huge number of symbols---essentially, all of
the symbols defined by \pkgname{latexsym}, \TC, the
various \AMS\ symbol sets, and \latexE\ itself.  The \TX\ and \PX\
conflicts are not listed in Table~\ref{name-clashes} because they are
designed to be compatible with the symbols they replace.
Table~\vref{benign-clash} illustrates what ``compatible'' means in this
context.

\begin{nonsymtable}{Example of a Benign Name Clash}
\label{benign-clash}
\begin{tabular}{@{}lcc@{}} \toprule
& Default & \TX \\
\multicolumn{1}{c}{\raisebox{1ex}[0pt][0pt]{Symbol}} & (Computer Modern) &
(Times Roman) \\ \cmidrule(r){1-1}\cmidrule(l){2-3}
\texttt{R} & \Huge R & {\fontfamily{txr}\selectfont \Huge R} \\
\cmd{\textrecipe} & \Huge\textrecipe &
  {\fontfamily{txr}\selectfont \Huge\textrecipe} \\
\bottomrule
\end{tabular}
\end{nonsymtable}

To use the new \TXPX\ symbols without altering the document's main font,
merely reset the default font families back to their original values
after loading one of those packages:

\begin{verbatim}
    \renewcommand\rmdefault{cmr}
    \renewcommand\sfdefault{cmss}
    \renewcommand\ttdefault{cmtt}
\end{verbatim}

\fi   % TX test


\subsection{Where can I find the symbol for~$\ldots$~?}
\label{combining-symbols}

% Index this every time we use it.
\newcommand{\fontdefdtx}{%
  \texttt{fontdef.dtx}\index{fontdef.dtx=\texttt{fontdef.dtx}}\xspace}

If you can't find some symbol you're looking for in this document, there
are a few possible explanations:

\begin{itemize}
  \item The symbol isn't intuitively named.  As a few examples, the
  command to draw dice\index{dice} is ``\cmd{\Cube}''; a plus sign with
  a circle around it (``exclusive or''\index{exclusive or} to computer
  engineers) is ``\cmd{\oplus}''; and lightning bolts in fonts designed
  by German speakers may have ``blitz'' in their names.  The moral of
  the story is to be creative with synonyms when searching the index.

  \item The symbol is defined by some package that I overlooked (or
  deemed unimportant).  If there's some symbol package that you think
  should be included in the \doctitle, please send me e-mail at the
  address listed on the title page.

  \item The symbol isn't defined in any package whatsoever.
\end{itemize}

\ifPI
  Even in the last case, all is not lost.  Sometimes, a symbol exists
  in a font, but there is no \latex{} binding for it.  For example,
  the PostScript\index{PostScript} Symbol\index{Symbol font} font
  contains a ``\Pisymbol{psy}{191}''\index{arrows} symbol, which may
  be useful for representing a carriage\index{carriage return} return,
  but there is no package for accessing that symbol (as far as I
  know).  To produce an unnamed symbol, you need to switch to the font
  explicitly with \latexE's low-level font commands~\cite{fntguide}
  and use \cmd{\char} to request a specific character number in the
  font.\footnote{\pkgname{pifont} defines a convenient \cmd{\Pisymbol}
  command for accessing symbols in PostScript\index{PostScript} fonts
  by number.  For example,
  ``\cmd{\Pisymbol}\texttt{\string{psy\string}\string{191\string}}''
  produces ``\Pisymbol{psy}{191}''.}

  Symbols that do not exist in any font can sometimes be fabricated out of
  existing symbols.  The \latexE{} source file called \fontdefdtx contains
  a number of such definitions.  For example, \cmd{\models} (see
  Table~\vref{rel}) is defined in that file with:
\else
  Even in the last case, all is not lost.  Sometimes, a symbol exists in a
  font, but there is no \latex{} binding for it.  Or, if a symbol does not
  exist in any font, it may be possible to fabricate it out of existing
  symbols.  The \latexE{} source file called \fontdefdtx contains a number
  of such definitions.  For example, \cmd{\models} (see Table~\vref{rel})
  is defined in that file with:
\fi    % PI test

\begin{verbatim}
    \def\models{\mathrel|\joinrel=}
\end{verbatim}

\noindent
where \cmd{\mathrel} and \cmd{\joinrel} are used to control the horizontal
spacing.  (See \TeXbook for more information on those commands.)

With some simple pattern-matching, one can easily define a backward
\cmd{\models} sign (``$=\joinrel\mathrel|$''):

\begin{verbatim}
    \def\ismodeledby{=\joinrel\mathrel|}
\end{verbatim}
\indexcommand{\ismodeledby}

As another example, \fontdefdtx composes the \cmd{\ddots} symbol (see
Table~\vref{ord}) out of three periods, raised~7\,pt., 4\,pt., and
1\,pt., respectively:

\begin{verbatim}
    \def\ddots{\mathinner{\mkern1mu\raise7\p@
        \vbox{\kern7\p@\hbox{.}}\mkern2mu
        \raise4\p@\hbox{.}\mkern2mu\raise\p@\hbox{.}\mkern1mu}}
\end{verbatim}

\noindent
\cmd{\p@} is a \latexE{} shortcut for ``\texttt{pt}'' or
``\texttt{1.0pt}''.  The remaining commands are defined in \TeXbook.
To\label{revddots} draw a version of \cmd{\ddots} with the dots going
along the opposite diagonal, we merely have to reorder the
\verb|\raise7\p@|, \verb|\raise4\p@|, and \verb|\raise\p@|:

\begin{verbatim}
    \makeatletter
    \def\revddots{\mathinner{\mkern1mu\raise\p@
        \vbox{\kern7\p@\hbox{.}}\mkern2mu
        \raise4\p@\hbox{.}\mkern2mu\raise7\p@\hbox{.}\mkern1mu}}
    \makeatother
\end{verbatim}
\indexcommand{\revddots}

\noindent
(The \cmd{\makeatletter} and \cmd{\makeatother} commands are needed to
coerce \latex{} into accepting ``\texttt{@}'' as part of a macro name.)

\index{integrals|(}
As\label{dashint} a final example of creating new symbols out of
existing ones, the following code defines a principal value integral
symbol, which is an integral sign with a line through it:

\begin{verbatim}
    \def\Xint#1{\mathchoice
       {\XXint\displaystyle\textstyle{#1}}%
       {\XXint\textstyle\scriptstyle{#1}}%
       {\XXint\scriptstyle\scriptscriptstyle{#1}}%
       {\XXint\scriptscriptstyle\scriptscriptstyle{#1}}%
       \!\int}
    \def\XXint#1#2#3{{\setbox0=\hbox{$#1{#2#3}{\int}$}
         \vcenter{\hbox{$#2#3$}}\kern-.5\wd0}}
    \def\ddashint{\Xint=}
    \def\dashint{\Xint-}
\end{verbatim}

\noindent
\cmd{\dashint} produces a single-dashed integral sign
(``$\dashint$''), while \cmd{\ddashint} produces a double-dashed one
(``$\ddashint$'').  The same technique can be used to produce, for
example, clockwise and counterclockwise contour integrals.  (Search
the \texttt{comp.text.tex}\index{comp.text.tex=\texttt{comp.text.tex}}
archives for a post by Donald\index{Arseneau, Donald} Arseneau that
says exactly how.)  The preceding code was taken verbatim from the UK
\TeX{} Users' Group FAQ (\url{http://www.tex.ac.uk/faq}).
\index{integrals|)}

\bigskip

\ifx\diatop\undefined
% The following was copied verbatim from ipa.sty, from the wsuipa package.
\def\diatop[#1|#2]{%
    {\setbox1=\hbox{#1{}}% diacritic mark
     \setbox2=\hbox{#2{}}%  letter (the group {} in case it is a diacritic)
     \dimen0=\ifdim\wd1>\wd2\wd1\else\wd2\fi% compute the max width
        % the `natural height' of diacritics is 1ex;
        % \dimen1 is the shift upwards
     \dimen1=\ht2\advance\dimen1by-1ex%
        % center the diacritic mark on the width of the letter:
     \setbox1=\hbox to\dimen0{\hss#1{}\hss}%
     \leavevmode % force horizontal mode
     \rlap{\raise\dimen1\box1}% the raised diacritic mark
     \hbox to\dimen0{\hss#2\hss}% the letter
    }%
  }%
\fi

\index{accents|(}
\index{accents>multiple per character}
\index{multiple accents per character}
Accents\label{multiple-accents} are a special case of combining
existing symbols to make new symbols.  While various tables in this
document show how to add an accent to an existing symbol, some
applications, such as transliteration from non-Latin alphabets,
require \emph{multiple} accents per character.  For instance, the
creator of pdf\TeX\ writes his name as ``H\`an Th\diatop[\'|\^e]
Th\`anh''.  The \pkgname{wsuipa} package defines \cmd{\diatop} and
\cmd{\diaunder} macros for putting one or more diacritics or accents
above or below a given character.
\ifTIPA\ifWIPA
  For example,
  \verb+\diaunder[{\diatop[\'|\=]}|+\linebreak[0]\verb+\textsubdot{r}]+
  produces ``\diaunder[{\diatop[\'|\=]}|\textsubdot{r}]''.
\fi\fi
See the \pkgname{wsuipa} documentation for more information.

\index{accents>any character as}
The \pkgname{accents} package facilitates the fabrication of accents
in math mode.  Its \cmd{\accentset} command enables \emph{any}
character to be used as an accent.
\ifACCENTS
  For instance, \cmd{\accentset}\verb+{+\cmd{\star}\verb+}{f}+
  produces ``$\accentset{\star}{f}\,$'' and
  \cmd{\accentset}\verb+{e}{X}+ produces ``$\accentset{e}{X}$''.
\fi
\cmd{\underaccent} does the same thing, but places the accent beneath
the character.
\ifACCENTS
  This enables constructs like
  \cmd{\underaccent}\verb+{+\cmd{\tilde}\verb+}{V}+, which produces
  ``$\underaccent{\tilde}{V}$''.
\fi
\pkgname{accents} provides other accent-related features as well; see
the documentation for more information.
\index{accents|)}


\subsection{Math-mode spacing}
\label{math-spacing}

Terms such as ``binary operators'', ``relations'', and ``punctuation''
in Section~\ref{math-symbols} primarily regard the surrounding spacing.
(See the Short Math Guide for \latex~\cite{Downes:smg} for a nice
exposition on the subject.)  To use an symbol for a different purpose,
you can use the \TeX{} commands \cmd{\mathord}, \cmd{\mathop},
\cmd{\mathbin}, \cmd{\mathrel}, \cmd{\mathopen}, \cmd{\mathclose}, and
\cmd{\mathpunct}.  For example, if you want to use \cmd{\downarrow} as a
variable (an ``ordinary'' symbol) instead of a delimiter, you can write
``\verb|$3 x + \mathord{\downarrow}$|'' to get the properly spaced ``$3
x + \mathord{\downarrow}$'' rather than the awkward-looking ``$3 x +
\downarrow$''.  See \TeXbook for more information.

The purpose of the ``log-like symbols'' in
\ifAMS
  Tables~\ref{log} and~\ref{ams-log}
\else
  Table~\ref{log}
\fi
is to provide the correct amount of spacing around and within
multiletter function names.  Table~\vref{log-spacing} contrasts the
output of the log-like symbols with various, na\"{\i}ve alternatives.
In addition to spacing, the log-like symbols also handle subscripts
properly.  For example, ``\verb|\max_{p \in P}|'' produces ``$\max_{p
\in P}$'' in text, but ``$\displaystyle\max_{p \in P}$'' as part of a
displayed formula.

\begin{nonsymtable}{Spacing Around/Within Log-like Symbols}
\label{log-spacing}
\setlength{\tabcolsep}{1em}
\begin{tabular}{@{}ll@{}} \toprule
\latex{} expression & Output \\ \midrule
\verb|$r \sin \theta$|       & $r \sin \theta$ (best) \\
\verb|$r sin \theta$|        & $r sin \theta$         \\
\verb|$r \mbox{sin} \theta$| & $r \mbox{sin} \theta$  \\
\bottomrule
\end{tabular}
\end{nonsymtable}


\subsection{ASCII and Latin~1 quick reference}
\label{ascii-quickref}

\index{ASCII|(}

Table~\vref{ascii-table} amalgamates data from various other tables in
this document into a convenient reference for \latexE typesetting of
ASCII characters, i.e., the characters available on a
typical\footnote{typical for the United States, at least} computer
keyboard.  The first two columns list the character's ASCII code in
decimal and hexadecimal.  The third column shows what the character
looks like.  The fourth column lists the \latexE command to typeset the
character as a text character.  And the fourth column lists the \latexE
command to typeset the character within a
\verb|\texttt{|$\ldots$\verb|}| command (or, more generally, when
\verb|\ttfamily| is in effect).

\index{ASCII|)}

\begin{nonsymtable}{\latexE ASCII Table}
  \index{ASCII>table}
  \label{ascii-table}
  % Define an equivalent of \vdots that's the height of a "9".
  \newlength{\digitheight}
  \settoheight{\digitheight}{9}
  \newcommand{\digitvdots}{\raisebox{-1.5pt}[\digitheight]{$\vdots$}}

  % Replace all glyphs in a row with vertical dots.
  \makeatletter
  \newcommand{\skipped}{%
    \settowidth{\@tempdima}{99} \makebox[\@tempdima]{\digitvdots} &
    \settowidth{\@tempdima}{99} \makebox[\@tempdima]{\digitvdots} &
    \digitvdots &
    \digitvdots &
    \digitvdots \\
  }
  \makeatother

  % Typesetting a symbol by prefixing it with a "\".
  \newcommand{\bscommand}[1]{#1 & \cmd{#1} & \cmd{#1}}

  \begin{tabular}[t]{@{}*2{>{\ttfamily}r}c*2{>{\ttfamily}l}l@{}} \\ \toprule
    \multicolumn{1}{@{}c}{Dec} &
    \multicolumn{1}{c}{Hex} &
    \multicolumn{1}{c}{Char} &
    \multicolumn{1}{c}{Body text} &
    \multicolumn{1}{c@{}}{\ttfamily\string\texttt} \\ \midrule

    33 & 21 & ! & ! & ! \\
    34 & 22 & {\fontencoding{T1}\selectfont\textquotedbl} &
      \string\textquotedbl & " \\      % Not available in OT1
    35 & 23 & \bscommand{\#} \\
    36 & 24 & \bscommand{\$} \\
    37 & 25 & \bscommand{\%} \\
    38 & 26 & \bscommand{\&} \\
    39 & 27 & ' & ' & ' \\
    40 & 28 & ( & ( & ( \\
    41 & 29 & ) & ) & ) \\
    42 & 2A & * & * & * \\
    43 & 2B & + & + & + \\
    44 & 2C & , & , & , \\
    45 & 2D & - & - & - \\
    46 & 2E & . & . & . \\
    47 & 2F & / & / & / \\
    48 & 30 & 0 & 0 & 0 \\
    49 & 31 & 1 & 1 & 1 \\
    50 & 32 & 2 & 2 & 2 \\
    \skipped
    57 & 39 & 9 & 9 & 9 \\
    58 & 3A & : & : & : \\
    59 & 3B & ; & ; & ; \\
    60 & 3C & \textless & \cmd{\textless} & < \\         % Or $<$
    61 & 3D & = & = & = \\ \bottomrule
  \end{tabular}
  \hfil
  \begin{tabular}[t]{@{}*2{>{\ttfamily}r}c*2{>{\ttfamily}l}l@{}} \\ \toprule
    \multicolumn{1}{@{}c}{Dec} &
    \multicolumn{1}{c}{Hex} &
    \multicolumn{1}{c}{Char} &
    \multicolumn{1}{c}{Body text} &
    \multicolumn{1}{c@{}}{\ttfamily\string\texttt} \\ \midrule

    62 & 3E & \textgreater & \cmd{\textgreater} & > \\   % Or $>$
    63 & 3F & ? & ? & ? \\
    64 & 40 & @ & @ & @ \\
    65 & 41 & A & A & A \\
    66 & 42 & B & B & B \\
    67 & 43 & C & C & C \\
    \skipped
    90 & 5A & Z & Z & Z \\
    91 & 5B & [ & [ & [ \\
    92 & 5C & \textbackslash & \cmd{\textbackslash} &
      \verb|\char`\\| \\   % \textbackslash works in non-OT1
    93 & 5D & ] & ] & ] \\
    94 & 5E & \^{} & \verb|\^{}| & \verb|\^{}| \\   % Or \textasciicircum
    95 & 5F & \_ & \verb|\_| & \verb|\char`\_| \\   % \_ works in non-OT1
    96 & 60 & ` & ` & ` \\
    97 & 61 & a & a & a \\
    98 & 62 & b & b & b \\
    99 & 63 & c & c & c \\
    \skipped
   122 & 7A & z & z & z \\
   123 & 7B & \{ & \verb|\{| & \verb|\char`\{| \\   % \{ works in non-OT1
   124 & 7C & \textbar & \cmd{\textbar} & | \\      % Or $|$
   125 & 7D & \} & \verb|\}| & \verb|\char`\}| \\   % \} works in non-OT1
   126 & 7E & \~{} & \verb|\~{}| & \verb|\~{}| \\   % Or \textasciitilde ($\sim$?)
   \\
   \bottomrule
  \end{tabular}
\end{nonsymtable}

The following are some additional notes about the contents of
Table~\ref{ascii-table}:

\begin{itemize}
  \item \cmd{\textquotedbl} is not available in the OT1 font encoding.

  \item The\label{upside-down} characters ``\texttt{<}'',
  ``\texttt{>}'', and ``\texttt{|}'' do work as expected in math mode,
  although they produce, respectively, ``<'', ``>'', and ``|'' in text
  mode.\footnote{Donald\index{Knuth, Donald E.} Knuth didn't think
  such symbols were important outside of mathematics, so he omitted
  them from the OT1 font encoding.}  Hence, \verb+$<$+, \verb+$>$+,
  and \verb+$|$+ serve as a terser alternative to \cmd{\textless},
  \cmd{\textgreater}, and \cmd{\textbar}.  Note that for typesetting
  metavariables, many people prefer \cmd{\textlangle} and
  \cmd{\textrangle} to \cmd{\textless} and \cmd{\textgreater}, i.e.,
  ``\meta{filename}'' instead of ``$<$\textit{filename}$>$''.

  \item The various \verb|\char| commands within \verb|\texttt| are
  necessary only in the OT1 font encoding.  Using other encodings
  (e.g.,~T1), commands such as \cmd{\{}, \cmd{\}}, \cmd{\_}, and
  \cmd{\textbackslash} all work properly.

  \item \label{tildes}\index{tildes} \cmd{\textasciicircum} can be
  used instead of \cmd{\^}\verb|{}|, and \cmd{\textasciitilde} can be
  used instead of \cmd{\~}\verb|{}|.  For typesetting tildes in URLs
  and Unix filenames, some people prefer \cmd{\sim} (see
  Table~\vref{rel}), which produces a larger symbol.  But if you don't
  mind the tilde produced by \cmd{\~}\verb|{}|, you should use the
  \pkgname{url} package to typeset URLs---it has a number of
  additional nice features.

  \item The IBM\index{IBM} version of ASCII\index{ASCII} characters~1
  to~31 can be typeset using the \pkgname{ascii} package.
\ifASCII
  See Table~\vref{ibm-ascii}.
\fi

  \item To replace~\verb|`| and~\verb|'| with the more computer-like
  (and more visibly distinct)~\texttt{\char18} and~\texttt{\char13}
  within a \texttt{verbatim} environment, use the \pkgname{upquote}
  package.  Outside of \texttt{verbatim}, you can use \verb|\char18|
  and \verb|\char13| to get the modified quote characters.  (The
  former is actually a grave accent.)
\end{itemize}

\index{Latin 1|(}

Similar to Table~\ref{ascii-table}, Table~\vref{latin1-table} is an
amalgamation of data from other tables in this document.  While
Table~\ref{ascii-table} shows how to typeset the 7-bit ASCII character
set, Table~\ref{latin1-table} shows the Latin~1 (Western European)
character set, also known as ISO-8859-1.

\index{Latin 1|)}

\begin{nonsymtable}{\latexE Latin~1 Table}
  \index{Latin 1>table}
  \label{latin1-table}

  \newcommand{\accented}[2]{#1#2 & \texttt{\string#1\string{#2\string}}}
  \newcommand{\encone}[1]{{\fontencoding{T1}\selectfont#1}}

  \begin{tabular}[t]{@{}*2{>{\ttfamily}r}c>{\ttfamily}lc@{}} \\ \toprule
    \multicolumn{1}{@{}c}{Dec} &
    \multicolumn{1}{c}{Hex} &
    \multicolumn{1}{c}{Char} &
    \multicolumn{2}{c@{}}{\latexE} \\ \midrule

    161 & A1 & !`                 & !{}` \\
    162 & A2 & \textcent          & \cmd{\textcent} & (\textsf{tc}) \\
    163 & A3 & \pounds            & \cmd{\pounds} \\
    164 & A4 & \textcurrency      & \cmd{\textcurrency} & (\textsf{tc}) \\
    165 & A5 & \textyen           & \cmd{\textyen} & (\textsf{tc}) \\
    166 & A6 & \textbrokenbar     & \cmd{\textbrokenbar} & (\textsf{tc}) \\
    167 & A7 & \S                 & \cmd{\S} \\
    168 & A8 & \textasciidieresis & \cmd{\textasciidieresis} & (\textsf{tc}) \\
    169 & A9 & \textcopyright     & \cmd{\textcopyright} \\
    170 & AA & \textordfeminine   & \cmd{\textordfeminine}   \\
    171 & AB & \encone{\guillemotleft} & \cmd{\guillemotleft} & (T1) \\
    172 & AC & \textlnot          & \cmd{\textlnot} & (\textsf{tc}) \\
    174 & AE & \textregistered    & \cmd{\textregistered} \\
    175 & AF & \textasciimacron   & \cmd{\textasciimacron} & (\textsf{tc}) \\
    176 & B0 & \textdegree        & \cmd{\textdegree} & (\textsf{tc}) \\
    177 & B1 & \textpm            & \cmd{\textpm} & (\textsf{tc}) \\
    178 & B2 & \texttwosuperior   & \cmd{\texttwosuperior} & (\textsf{tc}) \\
    179 & B3 & \textthreesuperior & \cmd{\textthreesuperior} & (\textsf{tc}) \\
    180 & B4 & \textasciiacute    & \cmd{\textasciiacute} & (\textsf{tc}) \\
    181 & B5 & \textmu            & \cmd{\textmu} & (\textsf{tc}) \\
    182 & B6 & \P                 & \cmd{\P} \\
    183 & B7 & \textperiodcentered & \cmd{\textperiodcentered} \\
    184 & B8 & \c{}               & \cmd{\c}\verb|{}| \\
    185 & B9 & \textonesuperior   & \cmd{\textonesuperior} & (\textsf{tc}) \\
    186 & BA & \textordmasculine  & \cmd{\textordmasculine} \\
    187 & BB & \encone{\guillemotright} & \cmd{\guillemotright} \\
    188 & BC & \textonequarter    & \cmd{\textonequarter} & (\textsf{tc}) \\
    189 & BD & \textonehalf       & \cmd{\textonehalf} & (\textsf{tc}) \\
    190 & BE & \textthreequarters & \cmd{\textthreequarters} & (\textsf{tc}) \\
    191 & BF & ?`                 & ?{}` \\
    192 & C0 & \accented{\`}{A} \\
    193 & C1 & \accented{\'}{A} \\
    194 & C2 & \accented{\^}{A} \\
    195 & C3 & \accented{\~}{A} \\
    196 & C4 & \accented{\"}{A} \\
    197 & C5 & \AA                & \string\AA \\
    198 & C6 & \AE                & \string\AE \\
    199 & C7 & \accented{\c}{C} \\
    200 & C8 & \accented{\`}{E} \\
    201 & C9 & \accented{\'}{E} \\
    202 & CA & \accented{\^}{E} \\
    203 & CB & \accented{\"}{E} \\
    204 & CC & \accented{\`}{I} \\
    205 & CD & \accented{\'}{I} \\
    206 & CE & \accented{\^}{I} \\
    207 & CF & \accented{\"}{I} \\
    208 & D0 & \encone{\DH}       & \string\DH & (T1) \\ \bottomrule
  \end{tabular}
  \hfil
  \begin{tabular}[t]{@{}*2{>{\ttfamily}r}c>{\ttfamily}lc@{}} \\ \toprule
    \multicolumn{1}{@{}c}{Dec} &
    \multicolumn{1}{c}{Hex} &
    \multicolumn{1}{c}{Char} &
    \multicolumn{2}{c@{}}{\latexE} \\ \midrule

    209 & D1 & \accented{\~}{N} \\
    210 & D2 & \accented{\`}{O} \\
    211 & D3 & \accented{\'}{O} \\
    212 & D4 & \accented{\^}{O} \\
    213 & D5 & \accented{\~}{O} \\
    214 & D6 & \accented{\"}{O} \\
    215 & D7 & \texttimes         & \string\texttimes & (\textsf{tc}) \\
    216 & D8 & \O                 & \string\O \\
    217 & D9 & \accented{\`}{U} \\
    218 & DA & \accented{\'}{U} \\
    219 & DB & \accented{\^}{U} \\
    220 & DC & \accented{\"}{U} \\
    221 & DD & \accented{\'}{Y} \\
    222 & DE & \encone{\TH}       & \string\TH & (T1) \\
    223 & DF & \ss                & \string\ss \\
    224 & E0 & \accented{\`}{a} \\
    225 & E1 & \accented{\'}{a} \\
    226 & E2 & \accented{\^}{a} \\
    227 & E3 & \accented{\~}{a} \\
    228 & E4 & \accented{\"}{a} \\
    229 & E5 & \aa                & \string\aa \\
    230 & E6 & \ae                & \string\ae \\
    231 & E7 & \accented{\c}{c} \\
    232 & E8 & \accented{\`}{e} \\
    233 & E9 & \accented{\'}{e} \\
    234 & EA & \accented{\^}{e} \\
    235 & EB & \accented{\"}{e} \\
    236 & EC & \accented{\`}{\i} \\
    237 & ED & \accented{\'}{\i} \\
    238 & EE & \accented{\^}{\i} \\
    239 & EF & \accented{\"}{\i} \\
    240 & F0 & \encone{\dh}       & \string\dh & (T1) \\
    241 & F1 & \accented{\~}{n} \\
    242 & F2 & \accented{\`}{o} \\
    243 & F3 & \accented{\'}{o} \\
    244 & F4 & \accented{\^}{o} \\
    245 & F5 & \accented{\~}{o} \\
    246 & F6 & \accented{\"}{o} \\
    247 & F7 & \textdiv           & \string\textdiv & (\textsf{tc}) \\
    248 & F8 & \o                 & \string\o \\
    249 & F9 & \accented{\`}{u} \\
    250 & FA & \accented{\'}{u} \\
    251 & FB & \accented{\^}{u} \\
    252 & FC & \accented{\"}{u} \\
    253 & FD & \accented{\'}{y} \\
    254 & FE & \encone{\th}       & \string\th & (T1) \\
    255 & FF & \accented{\"}{y} \\ \bottomrule
  \end{tabular}
\end{nonsymtable}

The following are some additional notes about the contents of
Table~\ref{latin1-table}:

\begin{itemize}
  \item A ``(\textsf{tc})'' after a symbol name means that the \TC\
  package must be loaded to access that symbol.  A ``(T1)'' means that
  the symbol needs the T1 font encoding.  The \pkgname{fontenc}
  package can change the font encoding document-wide.

  \item Many of the \verb|\text|\dots\ accents can also be produced
  using the accent commands shown in Table~\vref{text-accents} plus an
  empty argument.  For instance, \cmd{\=}\verb|{}| is essentially the
  same as \cmd{\textasciimacron}.

  \item The commands in the ``\latexE'' columns work in both body text
  and within a \verb|\texttt{|$\ldots$\verb|}| command (or, more
  generally, when \verb|\ttfamily| is in effect).

  \item \index{CP1252|(} Microsoft\textregistered{}\index{Microsoft
  Windows} Windows\textregistered{}\index{Windows} normally uses a
  superset of Latin~1 called ``CP1252'' (Code Page 1252).  CP1252 adds
  codes in the range~128--159 (hexadecimal~80--9F), including
  characters such as dashes, daggers, and quotation marks.  If there's
  sufficient interest, a future version of the \doctitle{} may include
  a CP1252 table.\index{CP1252|)}
\end{itemize}


\subsection{About this document}
\label{about-doc}

\paragraph{History}
David\index{Carlisle, David} Carlisle wrote the first version of this
document in October, 1994.  It originally contained all of the native
\latex{} symbols (Tables~\ref{bin}, \ref{op}, \ref{rel}, \ref{arrow},
\ref{log}, \ref{greek}, \ref{dels}, \ref{ldels}, \ref{math-accents},
\ref{other} \ref{punct}, and \ref{ord}) and was designed to be nearly
identical to the tables in Chapter~3 of Leslie\index{Lamport, Leslie}
Lamport's book~\cite{Lamport:latex}.  Even the table captions and the
order of the symbols within each table matched!  The \AMS\ symbols
(Tables~\ref{ams-bin}, \ref{ams-rel}, \ref{ams-nrel},
\ref{ams-arrows}, \ref{ams-narrows}, \ref{ams-greek},
\ref{ams-hebrew}, \ref{ams-del}, and \ref{ams-misc}) and an initial
Math Alphabets table (Table~\ref{alphabets}) were added thereafter.
Later, Alexander\index{Holt, Alexander} Holt provided the \ST\ tables
(Tables~\ref{st-bin}, \ref{st-large}, \ref{st-rel}, \ref{st-nrel},
\ref{st-arrows}, \ref{st-del}, and \ref{st-ext}).

In January, 2001, Scott\index{Pakin, Scott} Pakin took responsibility
for maintaining the symbol list and has since implemented a complete
overhaul of the document.  The result, now called, ``The \doctitle'',
includes the following new features:

\begin{itemize}
  \item The addition of a handful of new math alphabets, dozens of new
  font tables, and thousands of new symbols

  \item The categorization of the symbol tables into body-text
  symbols, mathematical symbols, science and technology symbols,
  dingbats, and other symbols, to provide a more user-friendly
  document structure

  \item An index, table of contents, and a frequently-requested symbol
  list, to help users quickly locate symbols

  \item Symbol tables rewritten to list the symbols in alphabetical
  order

  \item Appendices to provide additional information relevant to using
  symbols in \latex{}

  \item Tables showing how to typeset all of the characters in the
  ASCII\index{ASCII} and Latin~1\index{Latin 1} font encodings
\end{itemize}

\noindent
Furthermore, the internal structure of the document has been
completely altered from David's original version.  Most of the changes
are geared towards making the document easier to extend, modify, and
reformat.


\paragraph{Build characteristics}
Table~\vref{doc-characteristics} lists some of this document's build
characteristics.  Most important is the list of packages that \latex{}
couldn't find, but that \texttt{\jobname.tex} otherwise would have
been able to take advantage of.  Complete, prebuilt versions of this
document are available from CTAN\index{CTAN}
(\url{http://www.ctan.org/} or one of its many mirror sites) in the
directory \texttt{tex-archive/info/symbols/comprehensive}.

\begin{nonsymtable}{Document Characteristics}
\label{doc-characteristics}
\begin{tabular}{@{}lp{0.5\textwidth}@{}} \toprule
Characteristic      & Value \\ \midrule
Source file:        & \texttt{\jobname.tex} \\
Build date:         & \today \\
Symbols documented: & \prevtotalsymbols{} \\
Packages included:  & \makeatletter
                        \def\@elt#1{\pkgname{#1}\xspace}
                        \foundpkgs
                      \makeatother \\
Packages omitted:   & \makeatletter
                        \ifcomplete
                          \emph{none}
                        \else
                          \def\@elt#1{\pkgname{#1}\xspace}
                          \missingpkgs
                        \fi
                      \makeatother \\
\bottomrule
\end{tabular}
\end{nonsymtable}


% It seems like such a waste to put such a brief bibliography on its own
% page.  So we temporarily restore \section back to its original
% definition, just for the list of references.

\vfill
\begingroup
\samepage
\let\section=\origsection

\addcontentsline{toc}{section}{References}
\begin{thebibliography}{Dow00}
\bibitem[Dow00]{Downes:smg}
  Michael Downes.\index{Downes, Michael J.}
  Short math guide for {\latex},
  July~19, 2000.
  Version~1.07.
  Available from \url{http://www.ams.org/tex/short-math-guide.html}.

\bibitem[Knu86]{Knuth:ct-a}
  Donald~E. Knuth.\index{Knuth, Donald E.}
  \emph{The {\TeX}book},
  volume~A of \emph{Computers and Typesetting}.
  Ad{\-d}i{\-s}on-Wes{\-l}ey,
  Reading, MA, USA,
  1986.

\bibitem[Lam86]{Lamport:latex}
  Leslie Lamport.\index{Lamport, Leslie}
  \emph{\latex{}: A document preparation system}.
  Ad{\-d}i{\-s}on-Wes{\-l}ey,
  Reading, MA, USA,
  1986.

\bibitem[\LaT{}00]{fntguide}
  {\latex{}3 Project Team}.
  \latexE font selection,
  January~30, 2000.
  Available from
  \url{http://www.ctan.org/tex-archive/macros/latex/doc/fntguide.ps}
  (also included in many \TeX{} distributions).
\end{thebibliography}
\endgroup

\clearpage
\addcontentsline{toc}{section}{Index}
{\small\printindex}

\end{document}
