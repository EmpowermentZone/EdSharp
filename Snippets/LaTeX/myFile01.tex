\documentclass[11pt]{article}
\usepackage{graphicx}    % needed for including graphics e.g. EPS, PS
\topmargin -1.5cm        % read Lamport p.163
\oddsidemargin -0.04cm   % read Lamport p.163
\evensidemargin -0.04cm  % same as oddsidemargin but for left-hand pages
\textwidth 16.59cm
\textheight 21.94cm 
%\pagestyle{empty}       % Uncomment if don't want page numbers
\parskip 7.2pt           % sets spacing between paragraphs
%\renewcommand{\baselinestretch}{1.5} % Uncomment for 1.5 spacing between lines
\parindent 0pt		 % sets leading space for paragraphs

\begin{document}         
% Start your text
\section{Introduction}
\label{Introduction}

Latex provides several standard class styles.  In the
documentclass statement in top line of this file shows this
will be an article with a 11 point font.  The {\bf article} 
style is most suited for {\it conferences}.

\subsection{The Label Statement}
\label{labelStatement}

Section~\ref{Introduction} described the article style.  Notice
how Latex lets you use the label name to refer to a whatever
section or sub-section you desire?  Latex automatically generates
the numbers.

\subsection{Text Formatting}
\label{textFormatting}

Latex provides several standard text formatting modifiers.
For example, to bold a text, you surround the text in french
braces like {\bf this}.  Italics are also {\it possible}.
Lamport's book \cite{lamport} describes additional text formatting
commands.

\begin{thebibliography}{99}
\bibitem{lamport} Lamport, L., {\it LaTeX : A Documentation
 Preparation System User's Guide and Reference Manual}, Addison-Wesley 
 Pub Co., 2nd edition, August 1994.
\end{thebibliography}
 
% Stop your text
\end{document}

